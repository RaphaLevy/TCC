\chapter{Conclusão}

Ao longo do desenvolvimento do TCC, foram exploradas diferentes modificações às dinâmicas 
de transmissão da malária na Amazônia, de forma a aproximar a modelagem do mais compatível
com a história natural da doença nesse ambiente, com o objetivo final de entender
como impactos ecológicos na região afetam as interações entre vetor e hospedeiro.
\\\\
Com os resultados obtidos, foi possível
perceber o efeito que o maior contato entre humanos e mosquitos devido ao desmatamento
pode ter na dinâmica da malária com base na proporção de picadas causando infecção. 
Mais ainda, foi possível verificar como, dependendo dos parâmetros originais passados,
será necessária uma aproximação muito mais elevada entre vetor e hospedeiro para 
que a doença se torne endêmica na região amazônica. Como verificado, é possível que esse
contato chegue ao dobro do que é normalmente, e isso ainda não é suficiente 
para que a doença se torne uma epidemia.
\\\\
Para aproximar ainda mais os métodos usados aos comportamentos verificados na realidade,
poderia ser ideal a aplicação de um modelo de transmissão estocástico, incorporando as 
variáveis ambientais em constante mudança, mas para o que foi proposto, o modelo determinístico
utilizado foi suficiente para destacar as sensibilidades da doença às alterações 
climáticas e ambientais, permitindo uma análise clara e direcionada 
das interações entre vetor e hospedeiro, fornecendo uma base sólida 
para investigar as implicações das mudanças ambientais na transmissão da malária e para futuras investigações e aprimoramentos nos modelo.
