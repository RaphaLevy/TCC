\documentclass[12pt]{article}

\usepackage{setspace}
\usepackage{amsmath,amssymb}
\usepackage{amsfonts}
\usepackage{graphicx}
\usepackage[pdftex,bookmarks=true,bookmarksopen=false,bookmarksnumbered=true,colorlinks=true,linkcolor=black]{hyperref}
\usepackage[utf8]{inputenc}
\usepackage{float}
\usepackage{makecell}
\usepackage{tabularx}
\usepackage{changepage}
\usepackage{pbox}
\usepackage{pdfpages}
\usepackage{mwe}
%\usepackage[demo]{graphicx}
\usepackage{caption}
\usepackage{subcaption}
\usepackage{cellspace}
\usepackage[top=2.5cm]{geometry}
\setlength\cellspacetoplimit{4pt}
\setlength\cellspacebottomlimit{6pt}

\usepackage[brazil]{babel}
%\usepackage{pstricks}%, egameps}

%\setlength{\textwidth}{17.2cm}
%\setlength{\textheight}{23cm}
%\addtolength{\oddsidemargin}{-22mm} 
%\addtolength{\topmargin}{-15mm} \addtolength{\evensidemargin}{-15mm}
%\setlength{\parskip}{1mm}
%\setlength{\baselineskip}{500mm}

\newtheorem{theorem}{Theorem}[section]
\newtheorem{assumption}{Assumption}
\newtheorem{acknowledgment}{Acknowledgment}
\newtheorem{algorithm}{Algorithm}
\newtheorem{axiom}{Axiom}
\newtheorem{case}{Case}
\newtheorem{claim}{Claim}
\newtheorem{conclusion}{Conclusion} 
\newtheorem{condition}{Condition}
\newtheorem{conjecture}{Conjecture}
\newtheorem{corollary}{Corollary}[section]
\newtheorem{criterion}{Criterion}
\newtheorem{defn}{Definition}[section]

\newtheorem{example}{Example}[section]
\newtheorem{exercise}{Exercise}
\newtheorem{lemma}{Lemma}[section]
\newtheorem{notation}{Notation}
\newtheorem{problem}{Problem}
\newtheorem{proposition}{Proposition}[section]
\newtheorem{remark}{Remark}
\newtheorem{solution}{Solution}
\newtheorem{summary}{Summary}
\newenvironment{proof}[1][Proof]{\textbf{#1.} }{\rule{0.5em}{0.5em}}

\begin{document}

\begin{titlepage}
\begin{center}
\textbf{\LARGE Fundação Getulio Vargas}\\ 
\textbf{\LARGE Escola de Matemática Aplicada}\\
\textbf{\LARGE Curso de Graduação em Matemática Aplicada}

\par
\vspace{160pt}
\textbf{\Large Utilização de indicadores ambientais e epidemiológicos no estudo da dinâmica de doenças transmitidas por vetores}\\
\vspace{80pt}
\textbf{\Large por Raphael Felberg Levy}\\
\end{center}

\par
\vfill
\begin{center}
{{\normalsize Rio de Janeiro - Brasil}\\
{\normalsize \the\year}}
\end{center}
\end{titlepage}

\thispagestyle{empty}

\newpage
\begin{center}
\textbf{\LARGE Fundação Getulio Vargas}\\ 
\textbf{\LARGE Escola de Matemática Aplicada}\\
\textbf{\LARGE Curso de Graduação em Matemática Aplicada}

\par
\vspace{90pt}
\textbf{\Large Utilização de indicadores ambientais e epidemiológicos no estudo da dinâmica de doenças transmitidas por vetores}


\par
\vspace{65pt}
``\textbf{Declaro ser o único autor do presente projeto de
monografia que refere-se ao plano de trabalho a ser executado para continuidade da monografia e ressalto que não recorri a qualquer forma de colaboração ou auxílio de terceiros para realizá-lo a não ser nos casos e para os fins autorizados pelo professor orientador.}''
\end{center}

\par
\vspace{65pt}
\begin{center}


\hrulefill

\vspace{5pt}
\textbf{\Large Raphael Felberg Levy}
\end{center}

\par
\vfill
\begin{center}
{{\normalsize Rio de Janeiro - Brasil}\\
{\normalsize \the\year}}
\end{center}

\thispagestyle{empty}

\newpage
\begin{center}
\textbf{\LARGE Fundação Getulio Vargas}\\ 
\textbf{\LARGE Escola de Matemática Aplicada}\\
\textbf{\LARGE Curso de Graduação em Matemática Aplicada}

\par
\vspace{90pt}
\textbf{\Large Utilização de indicadores ambientais e epidemiológicos no estudo da dinâmica de doenças transmitidas por vetores}


\par
\vspace{65pt}

``\textbf{Projeto de Monografia apresentado à Escola de Matemática Aplicada como requisito parcial para continuidade ao trabalho de monografia.}''
\end{center}

\par
\vspace{65pt}
\begin{center}

\textbf{Aprovado em } \makebox[30pt]{\hrulefill}\textbf{ de }\makebox[120pt]{\hrulefill}\textbf{ de }\makebox[50pt]{\hrulefill}
\\
\vspace{5pt}
\textbf{Grau atribuído ao Projeto de Monografia:} \makebox[30pt]{\hrulefill}\\
\end{center}


\par
\vspace{40pt}
\begin{center}

\hrulefill

\vspace{5pt}
\textbf{Professor Orientador: Flávio Codeço Coelho}\\
\textbf{Escola de Matemática Aplicada}\\
\textbf{Fundação Getúlio Vargas}
\end{center}

\thispagestyle{empty}


\newpage
\tableofcontents
\thispagestyle{empty}

\newpage
\section{Introdução}

A Amazônia é uma das maiores e mais biodiversas florestas tropicais do mundo, abrigando inúmeras espécies de plantas, animais e microrganismos, incluindo vetores e patógenos responsáveis pela transmissão de diversas doenças. Entre elas, uma das mais comuns é a malária, que é causadas por protozoários do gênero \textit{Plasmodium}, transmitidos pela picada da fêmea infectada do mosquito do gênero \textit{Anopheles}. Ela está presente em 22 países americanos, porém as áreas com maior risco de infecção estão localizadas na região amazônica, englobando nove países, e que representaram $68\%$ dos casos de infecção em 2011 $^{[1]}$. Apesar de ser muito comum nas Américas, a malária não é limitada a esse continente, sendo encontrada em países da África e Ásia, tendo resultado em mais de dois milhões de casos de infecção e  445 mil mortes ao redor do mundo em 2016 $^{[2]}$.    
\\\\
Notavelmente, a transmissão de doenças por vetores é intimamente relacionada à alterações ambientais que interferem no ecossistema dos organismos transmissores e dos organismos afetados. No caso da Amazônia, povoados agrícolas e agropecuários são uns dos fatores que mais favorecem a transmissão da doença, tanto pelo desmatamento que causam para seu estabelecimento, assim como o agrupamento de pessoas em ambientes próximos ao habitat do vetor $^{[3]}$. Além disso, outros fatores, como chuvas, queimadas e mineração, também são muito influentes na transmissão de doenças na região. Esses eventos resultam em perda de habitat, fragmentação de ecossistemas e alterações no clima, afetando a distribuição e abundância de vetores e hospedeiros, bem como a interação entre eles e os patógenos. Além disso, o crescimento populacional e a urbanização também têm um papel importante na disseminação de doenças, uma vez que aumentam a exposição dos seres humanos aos vetores e aos riscos de infecção.
\\\\
Diante desse contexto, este trabalho visa investigar a transmissão de doenças por vetores na Amazônia e analisar como os impactos ambientais influenciam a dinâmica de transmissão da malária, os fatores ecológicos e socioeconômicos que afetam essa disseminação e possíveis estratégias de prevenção e controle, tendo como referência principal o Projeto Trajetórias-Sinbiose, elaborado pela FIOCRUZ, um dataset incluindo indicadores ambientais, epidemiológicos, econômicos e socioeconômicos para todos os municípios da Amazônia Legal, analisando a relação espacial e temporal entre trajetórias econômicas ligadas à dinâmica dos sistemas agrários, sendo eles rurais de base familiar ou produção agrícola e de gado em larga escala, a disponibilidade de recursos naturais e o risco de doenças $^{[4]}$.
\\\\
% (Ao longo deste trabalho, serão abordados os seguintes tópicos: (1) uma revisão das principais doenças transmitidas por vetores na Amazônia e seus vetores e patógenos associados; (2) análise dos fatores ecológicos, climáticos e socioeconômicos que influenciam a transmissão de doenças; (3) discussão sobre os modelos epidemiológicos, incluindo adaptações aos modelos SIR e SEI, para avaliar o impacto das mudanças ambientais na transmissão de doenças; e (4) identificação de estratégias de prevenção e controle baseadas na compreensão da dinâmica de transmissão e nos desafios específicos da região amazônica.)

\section{Resumo}



\section{Metodologia}

Para a elaboração do trabalho, serão usados dados populacionais do dataset do Projeto Trajetórias e dados climáticos do Climate Data, e serão abordados métodos de transmissão de doenças baseados em equações diferenciais ordinárias, como o SIR, e, partindo de uma modelagem simples, serão incluídos os fenômenos ambientais, como desmatamento e queimada, para ver como modificações no ecossistema irão interferir no modelo elaborado previamente. Os cálculos computacionais foram realizados em ambiente SageMath, utilizando funções de integração numérica do Scipy para solução do método.

\subsection{Modelos}

Descrevendo primeiramente SIR $^{[5], [6]}$, que pode ser considerado a base de modelos que serão usados ao longo do projeto, este foi desenvolvido por W. O. Kermack e A. G. McKendrick em 1927, sendo um dos modelos mais usados para a modelagem de epidemias, levando em consideração três compartimentos:

\begin{align*}
    & S: \text{número de indivíduos suscetíveis} \\
    & I: \text{número de indivíduos infectados} \\
    & R: \text{número de indivíduos recuperados}
\end{align*}
\\
Nesse modelo, os indivíduos saudáveis na classe $S$ são suscetíveis ao contato com indivíduos da classe $I$, e são transferidos para esse compartimento caso contraiam a doença. Indivíduos infectados podem espalhar a doença por contato direto com indivíduos suscetíveis, mas também podem se tornar imunes ao longo do tempo, sendo transferidos para o compartimento $R$. Em geral, $R$ inclui o total de recuperados (imunes) e mortos em decorrência da doença, mas podemos assumir que o número de mortos é muito baixo em relação ao tamanho da população total, podendo ser ignorado. Consideramos também que indivíduos nessa categoria não voltarão a ser suscetíveis ou infecciosos.   
\\\\
Considerando uma epidemia em um espaço curto de tempo e que a doença não é fatal, podemos ignorar dinâmicas vitais de nascimento e morte. Com isso, podemos descrever o modelo SIR através do seguinte sistema de EDOs:

\begin{gather*}
\begin{cases}
\dfrac{dS}{dt} = -\dfrac{\beta SI}{N} \\
\\
\dfrac{dI}{dt} = \dfrac{\beta SI}{N} - \gamma I \\
\\
\dfrac{dR}{dt} = \gamma I
\end{cases}
\end{gather*}
\\
No modelo, $N(t) = S(t)+I(t)+R(t)$, ou seja, a população total no tempo $t$, enquanto que $\beta$ é a taxa de infecção e $\gamma$ é a taxa de recuperação. Dado que $S+I+R$ é sempre constante se ignorarmos nascimento e morte, temos $\dfrac{dS}{dt}+\dfrac{dI}{dt}+\dfrac{dR}{dt} = 0$. 
\\\\
Para que a doença possa se espalhar, é fácil ver que $\dfrac{dI}{dt} = \dfrac{\beta SI}{N} - \gamma I > 0$. Assim, $\dfrac{\beta SI}{N} > \gamma I \Rightarrow \dfrac{\beta S}{N} > \gamma$. Supondo que estamos no início da infeccção, dado que queremos ver como se espalha, $I$ será muito pequeno e $S \approx N$. Concluímos então que $\dfrac{\beta N}{N} > \gamma \Rightarrow \dfrac{\beta}{\gamma} > 1$. É possível derivar esse valor adimensionalizando o modelo: sejam $y^* = \dfrac{S}{N}, \ x^* = \dfrac{I}{N}, \ z^* = \dfrac{R}{N}$ e $t^*=\dfrac{t}{1/\gamma} = \gamma t$, de forma que $y^*+x^*+z^*=1$. Substituindo o sistema de EDOs acima utilizando esses valores:

\begin{gather*}
\begin{cases}
\dfrac{dS}{dt} = \dfrac{d(y^*N)}{d(t^*/\gamma)} = -\dfrac{\beta SI}{N} = -\dfrac{\beta(y^*N)(x^*N)}{N} = -\beta y^*Nx^* \\
\\
\dfrac{dI}{dt} = \dfrac{d(x^*N)}{d(t^*/\gamma)} = \dfrac{\beta SI}{N} - \gamma I = \dfrac{\beta(y^*N)(x^*N)}{N} -\gamma(x^*N) = \beta y^*Nx^* - \gamma x^*N \\
\\
\dfrac{dR}{dt} = \dfrac{d(z^*N)}{d(t^*/\gamma)} = \gamma I = \gamma(x^*N)
\end{cases}
\end{gather*}

Agora, cancelando os fatores $N$ e $\gamma$ em ambos os lados das equações:

\begin{gather*}
\begin{cases}
\dfrac{d(y^*)}{d(t^*)} = -\dfrac{\beta y^*x^*}{\gamma} \\
\\
\dfrac{d(x^*)}{d(t^*)} = \dfrac{\beta y^*x^*}{\gamma} - x^* \\
\\
\dfrac{d(z^*)}{d(t^*)} = x^*
\end{cases}
\end{gather*}
\\
Sendo assim temos um sistema dado apenas por $y^*$ e $x^*$ e o parâmetro $\dfrac{\beta}{\gamma}$, que podemos chamar de $R_0$.
\\\\
Como esse trabalho será focado principalmente na modelagem de malária, irei agora apresentar um dos primeiros modelos desenvolvidos especialmente para essa doença, por Sir Ronald Ross em 1911 $^{[7]}$, que usa duas EDOs distintas das apresentadas acima:

\begin{gather*}
\begin{cases}
\dfrac{dI}{dt} = bp'i\dfrac{N-I}{N} -aI\\
\\
\dfrac{di}{dt} = bp(n-i)\dfrac{I}{N} - mI
\end{cases}
\end{gather*}
\\
Nesse caso, $N$ é a população humana total, $I(t)$ é o número de humanos infectados no tempo $t$, $n$ é a população total de mosquitos, $i(t)$ é o número de mosquitos infectados no tempo $t$, $b$ é a taxa de picadas, $p$ é a probabilidade de transmissão do humano para o mosquito por picada, $p'$ é a probabilidade de transmissão do mosquito para o humano por picada, $a$ é a taxa de recuperação da infecção de um humano e $m$ é a taxa de mortalidade dos mosquitos. $bp'i\dfrac{N-I}{N}dt -aIdt$ representam respectivamente o número de novos humanos infectados e o número de humanos recuperados no intervalo $dt$, enquanto que $bp(n-i)\dfrac{I}{N}dt - mIdt$ representam respectivamente o número de novos mosquitos infectados e o número de mosquitos que morrem nesse intervalo de tempo, assumindo que a infeccção não interfere na taxa de mortalidade dos mosquitos.
\\\\
Para esse modelo, Ross discutiu dois pontos de equilíbrio, em que $\dfrac{dI}{dt} = \dfrac{di}{dt} = 0$. Eles ocorrem quando $I=i=0$, que é o caso onde não existe malária, e, para $I, i > 0$, $I = N\dfrac{1-amN/(b^2pp'n)}{1+aN/(bp'n)}$ e $i = n\dfrac{1-amN/(b^2pp'n)}{1+m/(bp)}$. Ainda, para que a doença se estabeleça, $n$ deve ser maior que um valor limiar $n^* = \dfrac{amN}{b^2pp'}$. Nesse caso a doença se torna endêmica. Caso $n<n^*$, o equilíbrio estará em $I=i=0$ e a dença irá desaparecer.
\\\\
Dividindo as equações dos pontos de equilíbrio por $I \times i$, temos:

\begin{gather*}
\begin{cases}
\dfrac{bp}{N} = \dfrac{bpn}{Ni} -\dfrac{m}{I} \\
\\
\dfrac{bp'}{N} = \dfrac{bp'}{I} -\dfrac{a}{i} 
\end{cases}
\end{gather*}
\\
O que transforma o problema em um sistema linear com dois desconhecidos, $I$ e $i$.
\\\\
Agora, irei apresentar o modelo que será usado para o desenvolvimento do trabalho, feito com base no elaborado por Paul E. Parham e Edwin Michael em 2010, que leva em consideração fatores como a chuva e temperatura ($R$ e $T$, respectivamente) $^{[8]}$. 
\\\\
Definindo as equações que serão utilizadas:
\begin{gather*}
\begin{cases}
\dfrac{dS_H}{dt} = -ab_2\bigg(\dfrac{I_M}{N}\bigg)S_H\\
\\
\dfrac{dI_H}{dt} = ab_2\bigg(\dfrac{I_M}{N}\bigg)S_H-\gamma I_H\\
\\
\dfrac{dR_H}{dt} = \gamma I_H\\
\\
\dfrac{dS_M}{dt} = b - ab_1\bigg(\dfrac{I_H}{N}\bigg)S_M - \mu S_M\\
\\
\dfrac{dE_M}{dt} = ab_1\bigg(\dfrac{I_H}{N}\bigg)S_M - \mu E_M - ab_1\bigg(\dfrac{I_H}{N}\bigg)S_Ml(\tau_M)\\
\\
\dfrac{dI_M}{dt} = ab_1\bigg(\dfrac{I_H}{N}\bigg)S_Ml(\tau_M) -\mu I_M
\end{cases}
\end{gather*}
\\\\
É preciso comentar que o modelo original utilizava $I_M(t-\tau)$ em $\dfrac{dI_H}{dt}$ e $I_H(t-\tau)$ em $\dfrac{dE_M}{dt}$ (na passagem de $E$ para $I$) e $\dfrac{dI_M}{dt}$, respectivamente, mas como isso faria com que o modelo fosse baseado em equações com atraso, foi recomendado pelo orientador do Trabalho que essa diferença fosse desconsiderada, e usasse apenas $t$ atual.
\\\\
Tendo as equações do modelo para a população de humanos e de mosquitos, irei primeiro definir os parâmetros utilizados na modelagem e outras funções necessárias, e depois as variáveis usadas:
\\
\begin{adjustwidth}{-2cm}{}
\begin{center}
\renewcommand{\arraystretch}{1.5}
\raggedleft\begin{tabular}{|c | l | c|} 
 \hline
 \raisebox{-1ex}{\textbf{Parâmetro}} & \raisebox{-1ex}{\textbf{Definição}} & \raisebox{-1ex}{\textbf{Cálculo}}\\ 
 \hline
 $T(t)$ & \pbox{8cm}{\rule{0pt}{4.5ex}Temperatura\rule[-2.5ex]{0pt}{0pt}} & $T_1 (1 + T_2 \cos(\omega_1t - \phi_1))$\\ 
 \hline
 $R(t)$ & \pbox{8cm}{\rule{0pt}{4.5ex}Precipitação\rule[-2.5ex]{0pt}{0pt}} & $R_1 (1 + R_2 \cos(\omega_2t - \phi_2))$ \\
 \hline
 $b(R, T)$ & \pbox{8cm}{\rule{0pt}{4.5ex}Taxa de nascimento de mosquitos (/ dia)\rule[-2.5ex]{0pt}{0pt}} & $\dfrac{B_E  p_E(R)  p_L(R,T)  p_P(R)}{(\tau_E + \tau_L(T) + \tau_P)}$\\ 
 \hline
 $a(T)$ & \pbox{8cm}{\rule{0pt}{4.5ex}Taxa de picadas (/dia)\rule[-2.5ex]{0pt}{0pt}} & $\dfrac{(T - T_1)}{D_1}$ \\
 \hline
 $\mu(T)$ & \pbox{8cm}{\rule{0pt}{3ex}Taxa de mortalidade de mosquitos per capita (/ dia)\rule[-1.5ex]{0pt}{0pt}} & $-\log(p(T))$ \\
 \hline
 $\tau_M(T)$ & \pbox{8cm}{\rule{0pt}{4.5ex}Duração do ciclo de esporozoitos (dias)\rule[-2.5ex]{0pt}{0pt}} & $\dfrac{DD}{(T - T_{min})}$ \\
 \hline
 $\tau_L(T)$ & \pbox{8cm}{\rule{0pt}{4.5ex}Duração da fase de desenvolvimento das larvas (dias)\rule[-2.5ex]{0pt}{0pt}} & $\dfrac{1}{c_1T + c_2}$ \\
 \hline
 $p(T)$ & \pbox{8cm}{\rule{0pt}{3ex}Taxa diária de sobrevivência dos mosquitos \rule[-1.5ex]{0pt}{0pt}} & $e^{(-1 / (AT^2 + BT + C))}$ \\
 \hline
 $p_L(R)$ & \pbox{8cm}{\rule{0pt}{3ex}Probabilidade de sobrevivência das larvas dependente de chuva\rule[-1.5ex]{0pt}{0pt}} & $(\dfrac{4p_{ML}}{R_L^2})R(R_L - R)$ \\
 \hline
 $p_L(T)$ & \pbox{8cm}{\rule{0pt}{3ex}Probabilidade de sobrevivência das larvas dependente de temperatura\rule[-1.5ex]{0pt}{0pt}} & $e^{-(c_1T + c_2)}$ \\
 \hline
 $p_L(R, T)$ & \pbox{8cm}{\rule{0pt}{3ex}Probabilidade de sobrevivência das larvas dependente de temperatura e chuva\rule[-1.5ex]{0pt}{0pt}} & $p_L(R)p_L(T)$ \\
 \hline
 $l(\tau_M)(T)$ & \pbox{8cm}{\rule{0pt}{3ex}Probabilidade de sobrevivência de mosquitos durante o ciclo de esporozoitos (/ dia)\rule[-1.5ex]{0pt}{0pt}} & $p(T)^{\tau_M(T)}$ \\
 \hline
 $M(t)$ & \pbox{8cm}{\rule{0pt}{3ex}Número total de mosquitos\rule[-1.5ex]{0pt}{0pt}} & $S_M(t) + E_M(t) + I_M(t)$ \\
 \hline
 $N(t)$ & \pbox{8cm}{\rule{0pt}{3ex}Número total de humanos\rule[-1.5ex]{0pt}{0pt}} & $S_H(t) + I_H(t) + R_H(t)$ \\  
 \hline
\end{tabular}
\end{center}
\end{adjustwidth}

\begin{adjustwidth}{-0.5cm}{}
\begin{center}
\renewcommand{\arraystretch}{1.5}
\raggedleft\begin{tabular}{|c | c|} 
 \hline
 \textbf{Parâmetro} & \textbf{Definição}\\ 
 \hline
 $b_1$ & \makecell[l]{\rule{0pt}{3ex}Proporção de picadas de mosquitos suscetíveis \\ em humanos infectados que produzem infecção\rule[-1.5ex]{0pt}{0pt}} \\
 \hline
 $b_2$ & \makecell[l]{\rule{0pt}{3ex}Proporção de picadas de mosquitos infectados \\ em humanos suscetíveis que produzem infecção\rule[-1.5ex]{0pt}{0pt}} \\
 \hline
 $\gamma$ & \makecell[l]{\rule{0pt}{3ex}1/Duração média da infecciosidade em humanos (dias$^{-1}$)\rule[-1.5ex]{0pt}{0pt}} \\
 \hline
 $T_1$ & \makecell[l]{\rule{0pt}{3ex}Temperatura média na ausência de sazonalidade ($^\circ C$)\rule[-1.5ex]{0pt}{0pt}} \\
 \hline
 $T_2$ & \makecell[l]{\rule{0pt}{3ex}Amplitude da variabilidade sazonal na temperatura\rule[-1.5ex]{0pt}{0pt}} \\
 \hline
 $R_1$ & \makecell[l]{\rule{0pt}{3ex}Precipitação mensal média na ausência de \\ sazonalidade (mm)\rule[-1.5ex]{0pt}{0pt}} \\
 \hline
 $R_2$ & \makecell[l]{\rule{0pt}{3ex}Amplitude da variabilidade sazonal na precipitação\rule[-1.5ex]{0pt}{0pt}} \\
 \hline
 $\omega_1$ & \makecell[l]{\rule{0pt}{3ex}Frequência angular das oscilações sazonais na temperatura (meses$^{-1}$)\rule[-1.5ex]{0pt}{0pt}} \\
 \hline
 $\omega_2$ & \makecell[l]{\rule{0pt}{3ex}Frequência angular das oscilações sazonais na precipitação (meses$^{-1}$)\rule[-1.5ex]{0pt}{0pt}} \\
 \hline
 $\phi_1$ & \makecell[l]{\rule{0pt}{3ex}``Phase lag" da variabilidade da temperatura (defasagem de fase)\rule[-1.5ex]{0pt}{0pt}} \\
 \hline
 $\phi_2$ & \makecell[l]{\rule{0pt}{3ex}``Phase lag" da variabilidade da precipitação (defasagem de fase)\rule[-1.5ex]{0pt}{0pt}} \\
 \hline
 $B_E$ & \makecell[l]{\rule{0pt}{3ex}Número de ovos colocados por adulto por oviposição\rule[-1.5ex]{0pt}{0pt}} \\
 \hline
 $p_{ME}$ & \makecell[l]{\rule{0pt}{3ex}Probabilidade máxima de sobrevivência dos ovos\rule[-1.5ex]{0pt}{0pt}} \\
 \hline
 $p_{ML}$ & \makecell[l]{\rule{0pt}{3ex}Probabilidade máxima de sobrevivência das larvas\rule[-1.5ex]{0pt}{0pt}} \\
 \hline
 $p_{MP}$ & \makecell[l]{\rule{0pt}{3ex}Probabilidade máxima de sobrevivência das pupas\rule[-1.5ex]{0pt}{0pt}} \\
 \hline
 $\tau_E$ & \makecell[l]{\rule{0pt}{3ex}Duração da fase de desenvolvimento dos ovos (dias)\rule[-1.5ex]{0pt}{0pt}} \\
 \hline
 $b_3^*$ & \makecell[l]{\rule{0pt}{3ex}Taxa de infecção em mosquitos expostos $(1/\tau_M(T))$\rule[-1.5ex]{0pt}{0pt}} \\
 \hline
\end{tabular}
\end{center}
\end{adjustwidth}


\begin{adjustwidth}{-0.5cm}{}
\begin{center}
\renewcommand{\arraystretch}{1.5}
\raggedleft\begin{tabular}{|c | c|} 
 \hline
 \textbf{Parâmetro} & \textbf{Definição}\\ 
 \hline
  $\tau_P$ & \makecell[l]{\rule{0pt}{3ex}Duração da fase de desenvolvimento das pupas (dias)\rule[-1.5ex]{0pt}{0pt}} \\
 \hline
 $R_L$ & \makecell[l]{\rule{0pt}{3ex}Chuva limite até que os sítios de reprodução sejam eliminados, \\ removendo indivíduos de estágio imaturo (mm)\rule[-1.5ex]{0pt}{0pt}} \\
 \hline
 $T_{min}$ & \makecell[l]{\rule{0pt}{3ex}Temperatura mínima, abaixo dessa temperatura não há desenvolvimento \\ do parasita: 14.5 ($^\circ C$)\rule[-1.5ex]{0pt}{0pt}} \\
 \hline
 $DD$ & \makecell[l]{\rule{0pt}{3ex}``Degree days" para desenvolvimento do parasita. Número de graus em \\ que a temperatura média diária excede a temperatura mínima \\ de desenvolvimento.
 ``Sum of heat" para maturação: 105 $(^\circ C \ \text{dias})$ $^{[9], [12]}$\rule[-1.5ex]{0pt}{0pt}} \\
 \hline
 $A$ & \makecell[l]{\rule{0pt}{3ex}Parâmetro empírico de sensibilidade ($^\circ C^2 \ \text{dias}^{-1}$)\rule[-1.5ex]{0pt}{0pt}} \\
 \hline
 $B$ & \makecell[l]{\rule{0pt}{3ex}Parâmetro empírico de sensibilidade ($^\circ C \ \text{dias}^{-1}$)\rule[-1.5ex]{0pt}{0pt}} \\
 \hline
 $C$ & \makecell[l]{\rule{0pt}{3ex}Parâmetro empírico de sensibilidade ($\text{dias}^{-1}$)\rule[-1.5ex]{0pt}{0pt}} \\
 \hline
 $D_1$ & \makecell[l]{\rule{0pt}{3ex}Constante: 36.5 ($^\circ C \ \text{dias}$)\rule[-1.5ex]{0pt}{0pt}} \\
 \hline
 $c_1$ & \makecell[l]{\rule{0pt}{3ex}Parâmetro empírico de sensibilidade ($^\circ C \ \text{dias}^{-1}$)\rule[-1.5ex]{0pt}{0pt}} \\
 \hline
 $c_2$ & \makecell[l]{\rule{0pt}{3ex}Parâmetro empírico de sensibilidade ($\text{dias}^{-1}$)\rule[-1.5ex]{0pt}{0pt}} \\
 \hline
 $T'^*$ & \makecell[l]{\rule{0pt}{3ex}Parâmetro empírico de temperatura ($^\circ C$)\rule[-1.5ex]{0pt}{0pt}} \\
 \hline
\end{tabular}
\end{center}
\end{adjustwidth}

\vspace{1cm}
Parâmetros marcados com $^*$ foram adicionados durante o desenvolvimento da modelagem para correção de imprecisões derivadas das equações originais do artigo de referência.
\\\\
Com essas funções e parâmetros, podemos calcular o número reprodutivo básico ($R_0$) e os equilíbrios endêmicos:
\begin{gather*}
R_0 = \dfrac{Ma^2b_1b_2l(\tau_M)}{\gamma \mu N} \\
\\
I_M^* = \dfrac{M(R_0-1)}{(\dfrac{R_0}{l(\tau_M)}) + (\dfrac{ab_2M}{\gamma N})} \\
\\
I_H^* = \dfrac{N(R_0-1)}{R_0 + (\dfrac{ab_1}{\mu})}
\end{gather*}
Tendo as equações e parâmetros, a modelagem foi feita inicialmente utilizando dados da zona rural de Manaus, no período de 2004 a 2008, que foram selecionados devido à maior incidência de casos de malária causados por $P. \ vivax$. Usando a função de incidência utilizada no projeto Trajetorias $^{[4]}$, temos:
\begin{gather*}
    \text{Inc}(d, m, z, t_1, t_2) = \dfrac{\text{Casos}(d, m, z, t_1, t_2)}{\text{Pop}(m,z,(t_1+t_2)/2) \times 5 \ \text{anos}} \times 10^5,
\end{gather*}
onde $\text{casos}(d, m, z, t_1, t_2)$ é o número de casos da doença $d$ na zona $z$ do município $m$, e $t_1$ e $t_2$ são os anos iniciais e finais do intervalo, enquanto que $\text{pop}(m,z,(t_1+t_2)/2) \times 5 \ \text{anos}$ é a população na zona $z$ do município $m$ no meio do período multiplicado pelo total de anos de observação. Nesse caso, poderíamos indicar como:
\begin{gather*}
    \footnotesize{\text{Inc}(\text{Vivax}, \text{Manaus}, \text{Rural}, 2004, 2008) = \dfrac{\text{Casos}(\text{Vivax}, \text{Manaus}, \text{Rural}, 2004, 2008)}{\text{Pop}(\text{Manaus}, \text{Rural}, 2006) \times 5 \ \text{anos}} \times 10^5}  \\\\
    184030.8 = \dfrac{78745}{5\text{Pop}} \times 10^5 \Rightarrow Pop \approx 8558
\end{gather*}
Usando dados da população total de Manaus nesse período, cuja incidência foi de 3106.4 e o número de casos foi de 262264, a população total do município foi estimada como sendo de 1688540 habitantes. Com isso, a população rural pôde ser considerada como aproximadamente 0.5$\%$ da população do município.
\\\\
Tendo estimado o tamanho porcentual da população rural na cidade, foi possível calcular essa população em cada um dos anos da análise através de uma interpolação linear feita com dados de séries históricas do IBGE $^{[10]}$:
\\\\
\begin{adjustwidth}{-1cm}{}
\begin{center}
\renewcommand{\arraystretch}{1.5}
\begin{tabular}{|c | c|} 
 \hline
 \textbf{Ano} & \textbf{População rural estimada}\\ 
 \hline
$2004$ & $7717$ \\
 \hline
 $2005$ & $7889$ \\
 \hline
 $2006$ & $8061$ \\
 \hline
 $2007$ & $8233$ \\
 \hline
 $2008$ & $8492$ \\
 \hline
\end{tabular}
\end{center}
\end{adjustwidth}

\vspace{1cm}
Descrevendo agora um pouco da teoria por trás dos fatores ambientais, segundo $^{[13]}$, a remoção da copa das árvores permitiu a reemergência da malária na América do Sul, já que em áreas desmatadas, sem as copas cobrindo o solo, poças d'água sob luz solar atraem mosquitos da espécie $Anopheles \ darlingi$, principal vetor relacionado à malária humana na Amazônia $^{[14]}$, sendo que costumam ser menos encontrados em florestas ainda intactas. Isso ocorre porque a luz e calor favorecem o desenvolvimento de larvas e pupas, além de uma maior disponibilidade de algas para alimentação das larvas $^{[15]}$. O aumento da temperatura ambiente também favorece a capacidade vetorial dos mosquitos. O desmatamento também atrai e aproxima humanos para que possam tomar parte em atividades madeireiras, de agricultura e construção de rodovias, trazendo indíviduos infectados com o $Plasmodium$ para uma área que tanto o vetor quanto o ambiente já foram modificados de forma a favorecer a sua transmissão. Ainda, a agricultura também favorece à sedimentação dos rios, sendo ambientes propícios para o estabelecimento de criadouros. Sendo assim, pode ser considerado uma mudança adequada para o modelo para levar em consideração o desmatamento, o aumento das probabilidades de sobrevivência dos ovos, larvas e pupas, além de aumentar a proporção de picadas que produzem infecção, devido ao aumento da densidade populacional humana em áreas próximas aos criadouros dos mosquitos.
\newpage
\section{Resultados}

Analisando primeiro os resultados obtidos com os parâmetros originais do artigo de Parham e Michael ($T_1=23.2, \ T_2=0.07, \ \omega_1=0.67, \ \phi_1=1.53, \ R_1=85.9, \ R_2=0.98, \ \omega_2=0.65, \ \phi_2=1.99, \ A=-0.03, \ B=1.31, \ C=-4.4, \ b_1=0.04, \ b_2 = 0.09, \ T_{min}=14.5, \ \gamma= 1/120,\ R_L = 50, \ c_1=0.00554, \ c_2=-0.06737$), e utilizando a população média estimada anteriormente e um valor arbitrário para a população de mosquitos, de 10000, assumindo 1000 humanos infectados e 5000 mosquitos expostos à malária em $t=0$, a modelagem ficou como a seguir: 

% Pra plotar duas imagens uma ao lado da outra, precisa corrigir a posição das legendas.

% \begin{figure}
% \hspace*{-1.5cm} % Adiciona espaço negativo para puxar a imagem para a esquerda
% \begin{minipage}{.45\textwidth}
%   \centering
%   \includegraphics[width=1.25\linewidth]{SIR_Dados_Originais_Parham_Michael.png}
%   \captionof{figure}{A figure}
%   \label{fig:test1}
% \end{minipage}%
% \hspace{1.5cm} % Adiciona espaço horizontal
% \begin{minipage}{.45\textwidth}
%   \centering
%   \includegraphics[width=1.3\linewidth]{SEI_Dados_Originais_Parham_Michael.png}
%   \captionof{figure}{Another figure} % Legenda à direita da segunda imagem
%   \label{fig:test2}
% \end{minipage}
% \end{figure}





\begin{figure}[!ht]
        \centering
        \hbox{\hspace{2.0em} \includegraphics[scale=0.55] {SIR_Dados_Originais_Parham_Michael.png}}
        \caption{SIR com dados originais}
\end{figure} 
\begin{figure}[!ht]
        \centering
        \hbox{\hspace{2.0em} \includegraphics[scale=0.55] {SEI_Dados_Originais_Parham_Michael.png}}
        \caption{SEI com dados originais}
\end{figure} 
\newpage
A elaboração do modelo com os dados originais pode ser encontrada em
\\
https://github.com/RaphaLevy/TCC/blob/main/old/Dados\%20Originais.ipynb.
\\\\
Com essa modelagem inicial, é perceptível uma forte oscilação do número de humanos e mosquitos suscetíveis, assim como de humanos infectados. Ademais, é notável que, com esses parâmetros, a epidemia não irá se estabilizar, visto que o número de humanos infectados tende a 0 ao longo do ano, enquanto que a população de mosquitos suscetíveis fica negativa e a população de expostos e infectados também tende a 0. Esses efeitos foram caracterizados pela temperatura e precipitação oscilando em períodos de tempo muito curtos, devido a um alto valor de $\omega$ para ambas as funções. 
\\\\
Coletando dados climatológicos de Manaus em $^{[11]}$, a temperatura e precipitação média foram estimados como 26.4 $^\circ C$ e 250.083 mm, respectivamente. Com esses dados, a amplitude da variabilidade sazonal, frequência angular e ``phase lag" da variabilidade para ambos foram definidos de forma a aproximar os valores reais:

\begin{adjustwidth}{-2cm}{}
\begin{center}
\renewcommand{\arraystretch}{1.5}
\begin{tabular}{|c | c|} 
 \hline
 \textbf{Parâmetro} & \textbf{Valor}\\ 
 \hline
  $T_1$ & \makecell[l]{\rule{0pt}{3ex}26.4\rule[-1.5ex]{0pt}{0pt}} \\
 \hline
 $T_2$ & \makecell[l]{\rule{0pt}{3ex}0.025\rule[-1.5ex]{0pt}{0pt}} \\
 \hline
 $\omega_1$ & \makecell[l]{\rule{0pt}{3ex}0.017\rule[-1.5ex]{0pt}{0pt}} \\
 \hline
 $\phi_1$ & \makecell[l]{\rule{0pt}{3ex}-1.45\rule[-1.5ex]{0pt}{0pt}} \\
 \hline
 $R_1$ & \makecell[l]{\rule{0pt}{3ex}250.083\rule[-1.5ex]{0pt}{0pt}} \\
 \hline
 $R_2$ & \makecell[l]{\rule{0pt}{3ex}0.565\rule[-1.5ex]{0pt}{0pt}} \\
 \hline
 $\omega_2$ & \makecell[l]{\rule{0pt}{3ex}0.02\rule[-1.5ex]{0pt}{0pt}} \\
 \hline
 $\phi_2$ & \makecell[l]{\rule{0pt}{3ex}1.6\rule[-1.5ex]{0pt}{0pt}} \\
 \hline
\end{tabular}
\end{center}
\end{adjustwidth}

\vspace{1cm}
A temperatura e precipitação ao longo do ano evoluem então da seguinte maneira:


% \begin{figure}
% \hspace*{-1.5cm} % Adiciona espaço negativo para puxar a imagem para a esquerda
% \begin{minipage}{.45\textwidth}
%   \centering
%   \includegraphics[width=1.2\linewidth]{Grafico_da_Temperatura.png}
%   \captionof{figure}{A figure}
%   \label{fig:test1}
% \end{minipage}%
% \hspace{1.5cm} % Adiciona espaço horizontal
% \begin{minipage}{.45\textwidth}
%   \centering
%   \includegraphics[width=1.2\linewidth]{Grafico_da_Precipitacao.png}
%   \captionof{figure}{Another figure} % Legenda à direita da segunda imagem
%   \label{fig:test2}
% \end{minipage}
% \end{figure}


\begin{figure}[!ht]
        \centering
        \hbox{\hspace{7.0em} \includegraphics[scale=0.6] {Grafico_da_Temperatura.png}}
        \caption{Gráfico da temperatura}
\end{figure} 
\newpage
\begin{figure}[!ht]
        \centering
        \hbox{\hspace{7.0em} \includegraphics[scale=0.6] {Grafico_da_Precipitacao.png}}
        \caption{Gráfico da precipitação}
\end{figure} 
A elaboração dos gráficos acima pode ser encontrada em
\\
https://github.com/RaphaLevy/TCC/blob/main/Discuss\%C3\%A3o/
\\
Correcao\_de\_Modelagens.ipynb.
\\\\
De forma a garantir a corretude da função com os parâmetros utilizados, calculei os valores da temperatura nos meses de outubro e maio, que são o mais quente e frio do ano, com temperaturas médias de 27.6 $^\circ C$ e 25.8 $^\circ C$ respectivamente, e da precipitação em março e agosto, que são os meses com maior e menor precipitação, com 395 e 114 mm, respectivamente. Os valores médios obtidos foram de 27.06 $^\circ C$, 25.86 $^\circ C$, 390.67 mm e 112.89 mm. Tendo os parâmetros de $T$ e $R$ prontos para uma primeira análise, a evolução das populações de humanos e mosquitos foram verificadas utilizando $N=8558$, com $S_{H0} = 7558$, $I_{H0} = 10000$, $M= 300000$, $S_{M0}=250000$, $E_{M0} = 50000$, $A = 317.925, \ B = 15, \ C = -48.78$ e $R_L = 312 \ \text{mm}$:
\begin{figure}[!ht]
        \centering
        \hbox{\hspace{2.0em} \includegraphics[scale=0.55] {SIR_a_incorreto.png}}
        \caption{SIR original com $T'=26.4, \ \phi_1=-1.45$}
\end{figure} 
\newpage
\begin{figure}[!ht]
        \centering
        \hbox{\hspace{2.0em} \includegraphics[scale=0.55] {SEI_a_incorreto.png}}
        \caption{SEI original com $T'=26.4, \ \phi_1=-1.45$}
\end{figure} 
Notavelmente, essa modelagem está incorreta, especialmente a da população humana, cujo número de infectados decai para aproximadamente -300000. Verificando as EDOs, foi possível ver que o único parâmetro que é dado em função da temperatura e chuva no SIR é o $a(T)$, que é a taxa de picadas por dia. Dada a sua fórmula, e sabendo que a temperatura inicialmente decai nos primeiros meses do ano, tem-se que $a$ tomará valores negativos na equação de infectados, visto que $T_1$ será maior que $T$ inicialmente, explicando o comportamento da curva. Se comparado com os dados do artigo de Parham e Michael, cuja temperatura é crescente inicialmente, foi necessário modificar a ordem do numerador dessa taxa para que, num primeiro momento, não se torne negativa conforme a temperatura decai. Utilizando então a equação $a(T) = \dfrac{T_1-T}{D_1}$, os gráficos do modelo ficaram como a seguir:
\begin{figure}[!ht]
        \centering
        \hbox{\hspace{2.0em} \includegraphics[scale=0.55] {SIR_a_correto.png}}
        \caption{SIR atualizado com $T'=26.4, \ \phi_1=-1.45$}
\end{figure} 
\newpage
\begin{figure}[!ht]
        \centering
        \hbox{\hspace{2.0em} \includegraphics[scale=0.55] {SEI_a_correto.png}}
        \caption{SEI atualizado com $T'=26.4, \ \phi_1=-1.45$}
\end{figure}
A elaboração dos gráficos acima pode ser encontrada em
\\
https://github.com/RaphaLevy/TCC/blob/main/Discuss\%C3\%A3o/
\\
Testa\_Infectados\_Humanos.ipynb.
\\\\
Analisando o comportamento da modelagem de mosquitos de forma que a população ficasse aproximadamente constante ao longo do período, dadas as equações diferenciais do modelo SEI, e os parâmetros passados para atingir esse objetivo, o valor de $\mu$ passado fica muito próximo de 0, enquanto que $l(\tau_M)$, uma probabilidade, fica muito próxima de 1. Por isso, $\dfrac{dE_M}{dt}$ também fica bem próximo de 0, fazendo com que a função de expostos seja linear, aproximadamente constante no número inicial de infectados, enquanto que a população de mosquitos que sai do compartimento de suscetíveis quase que simultaneamente entra no compartimento de infectados, causando as ondulações espelhadas de $S$ e $I$. Para contornar esse efeito, foi necessário modificar o uso de $b_1$, para passar apenas mosquitos do compartimento $S$ para $E$, necessitando da inclusão de um novo parâmetro, $b_3$, para passar mosquitos do compartimento $E$ para $I$. Essa taxa é inversa ao período de incubação, então definimos
\begin{gather*}
    b_3 = \dfrac{T-T_{min}}{DD}
\end{gather*}
Ademais, foi removido o parâmetro $a(T)$ na passagem de mosquitos expostos para infectados, visto que nessa mudança do compartimento $E$ para $I$ não ocorrem novas picadas, assim como o parâmetro $T_1$ utilizado na fórmula de $a(T)$ também teve de ser modificado, visto que o artigo original aplica $T_1$ com dois valores diferentes, para a taxa de picadas e para a equação da temperatura por tempo. Com o intuito de manter a formatação de $T(t)$ e $R(t)$ a mesma, $T_1$ de $a$ foi modificado para $T'$.
\\\\
Considerando então o que foi dito acima, também se tornou necessário modificar as equações diferenciais do SEI, que ficaram dessa forma:
\begin{gather*}
\begin{cases}
\dfrac{dS_M}{dt} = b - ab_1\bigg(\dfrac{I_H}{N}\bigg)S_M - \mu S_M\\
\\
\dfrac{dE_M}{dt} = ab_1\bigg(\dfrac{I_H}{N}\bigg)S_M - \mu E_M - b_3E_Ml\\
\\
\dfrac{dI_M}{dt} = b_3E_Ml -\mu I_M\\
\end{cases}
\end{gather*}
Utilizando as equações atualizadas, dessa vez iniciando a população humana com nenhum infectado e usando $A, \ B, \ C$ e $R_L$ definidos previamente, os gráficos ficaram como a seguir:
\\\\
\begin{figure}[!ht]
        \centering
        \hbox{\hspace{2.0em} \includegraphics[scale=0.55] {SIR_Correcao_b3.png}}
        \caption{SIR com $T'=26.4, \ A=317.925, \ B=15, \ C=-48.78, \ R_L=312$}
\end{figure} 
\newpage
\begin{figure}[!ht]
        \centering
        \hbox{\hspace{2.0em} \includegraphics[scale=0.55] {SEI_Correcao_b3.png}}
        \caption{SEI com $T'=26.4, \ A=317.925, \ B=15, \ C=-48.78, \ R_L=312$}
\end{figure}
Com as adaptações feitas à transmissão da doença entre os mosquitos, é possível ver que não só o equilíbrio mudou, de forma que agora o número de expostos tende a 0, e não o de infectados, podemos ver também que a população de humanos é fortemente impactada, novamente oscilando constantemente, e com o número de infectados ficando negativo repetidas vezes. Analisando modificações nos parâmetros empíricos citados acima, foi notado que esse comportamento se dá devido em especial ao alto valor em módulo de $A$, visto que com $A=-217.925$, o comportamento do modelo foi similar. Até mesmo usando $A=17.925$, o comportamento observado foi similar. Contudo, para $A=0$, a população humana teve um comportamento bem mais viável, enquanto que a população de mosquitos tendeu à extinção:
\\\\
\begin{figure}[!ht]
        \centering
        \hbox{\hspace{2.0em} \includegraphics[scale=0.55] {SIR_Correcao_b3_A0.png}}
        \caption{SIR com $T'=26.4, \ A=0, \ B=15, \ C=-48.78, \ R_L=312$}
\end{figure} 
\begin{figure}[!ht]
        \centering
        \hbox{\hspace{2.0em} \includegraphics[scale=0.55] {SEI_Correcao_b3_A0.png}}
        \caption{SEI com $T'=26.4, \ A=0, \ B=15, \ C=-48.78, \ R_L=312$}
\end{figure}
\newpage
A elaboração dos gráficos acima pode ser encontrada em
\\
https://github.com/RaphaLevy/TCC/blob/main/Discuss\%C3\%A3o/
\\
Correcao\_de\_Modelagens\_2.ipynb.
\\\\
Analisando o efeito de $A$, notou-se que, para $A$ muito grande, $\mu$ se torna extremamente pequeno:
\begin{flalign*}
& \text{Para A} = 317.925, \ \mu = 4.505961269611858e-06 \\
& \text{Para A} = 17.925, \ \mu = 7.78802370175753e-05 \\
& \text{Para A} = -217.925, \ \mu = -6.599014101979344e-06 \\
& \text{Para A} = 0, \mu = 0.0028800184321179124
\end{flalign*}
\\
Com isso, a mortalidade de mosquitos será extremamente baixa para valores grandes de $A$, positivos ou não. Contudo, como uma forma de possibilitar o uso de valores altos para esse parâmetro, e ainda garantir que as população tenham valores sempre não-negativos, foi aplicado um máximo nas taxas com subtração, segundo recomendação do orientador, de forma que o valor resultante entre $T(t)-T_{min}$, $R_L - R(t)$ e $T'-T(t)$ seja o valor máximo entre essas diferenças e uma tolerância pequena, no caso foi utilizado um $\epsilon = 10^{-5}$. Isso foi suficiente para garantir que, mesmo com valores grandes de $A$, o modelo não tomasse valores negativos. Isso pode ser verificado nas imagens a seguir:
\begin{figure}[!ht]
        \centering
        \hbox{\hspace{4.0em} \includegraphics[scale=0.6] {SIR_Correcao_max_A317.png}}
        \caption{SIR com $T'=26.4, \ A=317.925, \ B=15, \ C=-48.78, \ R_L=312$}
\end{figure} 
\newpage
\begin{figure}[!ht]
        \centering
        \hbox{\hspace{3.5em} \includegraphics[scale=0.6] {SEI_Correcao_max_A317.png}}
        \caption{SEI com $T'=26.4, \ A=317.925, \ B=15, \ C=-48.78, \ R_L=312$}
\end{figure}
\begin{figure}[!ht]
        \centering
        \hbox{\hspace{4.0em} \includegraphics[scale=0.6] {SIR_Correcao_max_A0.png}}
        \caption{SIR com $T'=26.4, \ A=0, \ B=15, \ C=-48.78, \ R_L=312$}
\end{figure} 
\begin{figure}[!ht]
        \centering
        \hbox{\hspace{3.5em} \includegraphics[scale=0.6] {SEI_Correcao_max_A0.png}}
        \caption{SEI com $T'=26.4, \ A=0, \ B=15, \ C=-48.78, \ R_L=312$}
\end{figure}
\newpage
Com essas modificações, é possível notar como, independente do valor de $A$, a evolução das populações humanas será bem similar, com uma estabilização da população de suscetíveis logo antes de se tornar 0. A de mosquitos, por sua vez, é menos oscilante para valores grandes de $A$, e não se estabiliza mesmo em mais de 10000 dias. Por outro lado, se $A=0$, a população de mosquitos se aproxima da extinção, com pequenos picos de mosquitos unicamente suscetíveis, sem estabilização da doença. Contudo, ainda que usando o máximo entre um pequeno $\epsilon$ e as diferenças notadas acima, $T'=26.4$ é um valor menor que a temperatura máxima calculada para o ano, que é próxima de 27.1. Assim, foi recomendado pelo orientador que fosse testado um valor maior que esse máximo, e utilizando $T'=27.4$, os resultados obtidos foram:
\begin{figure}[!ht]
        \centering
        \hbox{\hspace{4.0em} \includegraphics[scale=0.6] {SIR_Correcao_max_T_linha.png}}
        \caption{SIR com $T'=27.4, \ A=317.925, \ B=15, \ C=-48.78, \ R_L=312$}
\end{figure} 
\begin{figure}[!ht]
        \centering
        \hbox{\hspace{3.5em} \includegraphics[scale=0.6] {SEI_Correcao_max_T_linha.png}}
        \caption{SEI com $T'=27.4, \ A=317.925, \ B=15, \ C=-48.78, \ R_L=312$}
\end{figure}
\\
Tendo então o modelo SIR/SEI devidamente corrigido, foi possível partir para as análises de aplicação do desmatamento. Para isso, iniciei com o cálculo do $\mathcal{R}_0$ para ambos SIR e SEI, e também para o modelo acoplado, usando como referência a formulação de P. van den Driessche $^{[16]}$.

\newpage
\section{Discussão}

Dado ao grande foco alocado na modelagem da transmissão analisando os impactos ambientais, e a necessidade constante de modificar o modelo proposto, tanto para corrigir estimações não-realistas, como o modelo oscilante apresentado nos primeiros gráficos do trabalho, quanto para alterar a metodologia usada no artigo de referência, em especial na modelagem dos mosquitos, a análise aprofundada dos efeitos de fatores socioeconômicos, assim como a identificação de estratégias de prevenção e controle da doença acabaram fugindo ao escopo do TCC.
\\\\
Dentro do que foi feito, a principal consideração que se pôde tirar do desenvolvimento do trabalho é que esse é um modelo muito sensível aos parâmetros utilizados. Pequenas modificações feitas em $A, \ B$ ou $C$ por exemplo são suficientes para que o modelo atinja um equilíbrio ou tenha as populações tendendo a $\pm \infty$, o que não faz sentido num ecossistema ``fechado". 

\newpage
\section{Conclusão}


\newpage

\section{Referências}
[1] Pimenta et al. An overview of malaria transmission from the perspective of Amazon $\text{Anopheles vectors}$. Mem Inst Oswaldo Cruz, Rio de Janeiro, Vol. 110(1): 23-47, February 2015. 
\\https://doi.org/10.1590/0074-02760140266.  
\\\\
\noindent [2] Josling, G. A., Williamson, K. C., Llinás, M. Regulation of Sexual Commitment and Gametocytogenesis in Malaria Parasites. Annual Review of Microbiology 2018 72:1, 501-519. https://doi.org/10.1146/annurev-micro-090817-062712. 
\\\\
\noindent [3] Silva-Nunes, M., Codeço, C. T. et al. Malaria on the Amazonian Frontier: Transmission Dynamics, Risk Factors, Spatial Distribution, and Prospects for Control. Am J Trop Med Hyg. 2008 Oct;79(4):624-35. PMID: 18840755.
\\\\
\noindent [4] Rorato, A.C., Dal’Asta, A.P., Lana, R.M. et al. Trajetorias: a dataset of environmental, epidemiological, and economic indicators for the Brazilian Amazon. Sci Data 10, 65 (2023). https://doi.org/10.1038/s41597-023-01962-1.
\\\\
\noindent [5] Coelho, F. C. Github Modelagem-Matematica-IV. \\ https://github.com/fccoelho/Modelagem-Matematica-IV/tree/master
\\\\
\noindent [6] Prasad, R., Sagar, S.K., Parveen, S. et al. Mathematical modeling in perspective of vector-borne viral infections: a review. Beni-Suef Univ J Basic Appl Sci 11, 102 (2022). https://doi.org/10.1186/s43088-022-00282-4.
\\\\
\noindent [7] Bacaër, N. (2011). Ross and malaria (1911). In: A Short History of Mathematical Population Dynamics. Springer, London. \\ https://doi.org/10.1007/978-0-85729-115-8$\_$12.
\\\\
\noindent [8] Parham, P.E., Michael, E. (2010). Modelling Climate Change and Malaria Transmission. In: Michael, E., Spear, R.C. (eds) Modelling Parasite Transmission and Control. Advances in Experimental Medicine and Biology, vol 673. Springer, New York, NY. https://doi.org/10.1007/978-1-4419-6064-1$\_$13.
\\\\
\noindent [9] McCord, G. Malaria ecology and climate change. Eur. Phys. J. Spec. Top. 225, 459–470 (2016). https://doi.org/10.1140/epjst/e2015-50097-1
\\\\
\noindent [10] Censo - Séries históricas. Brasil / Amazonas / Manaus. 
\\https://cidades.ibge.gov.br/brasil/am/manaus/pesquisa/43/0?tipo=grafico
\\\\
\noindent [11] Climate Data. CLIMA MANAUS (BRASIL). https://pt.climate-data.org \\ /america-do-sul/brasil/amazonas/manaus-1882/
\\\\
\noindent [12] Detinova, Tatiana Sergeevna, Bertram, D. S $\&$ World Health Organization. (1962). Age-grouping methods in diptera of medical importance, with special reference to some vectors of malaria / T. S. Detinova ; [with] an Annex on the ovary and ovarioles of mosquitos (with glossary) by D. S. Bertram. World Health Organization. \\ https://iris.who.int/handle/10665/41724
\\\\
\noindent [13] Norris, D.E. Mosquito-borne Diseases as a Consequence of Land Use Change. EcoHealth 1, 19–24 (2004). https://doi.org/10.1007/s10393-004-0008-7
\\\\
\noindent [14] Anopheles. Fiocruz Rondônia. \\ https://www.rondonia.fiocruz.br/pivem/anopheline/
\\\\
\noindent [15] Silva-Nunes, M. Impacto de Alterações Ambientais na Transmissão da Malária e Perspectivas para o Controle da Doença em Áreas de Assentamento Rural da Amazônia Brasileira. \\ https://revistas.ufrj.br/index.php/oa/article/view/7101/5685
\\\\
\noindent [16] van den Driessche P, Watmough J. Reproduction numbers and sub-threshold endemic equilibria for compartmental models of disease transmission. Math Biosci. 2002 Nov-Dec;180:29-48. doi: 10.1016/s0025-5564(02)00108-6. PMID: 12387915. \\ 
https://pubmed.ncbi.nlm.nih.gov/12387915/


\newpage
\section{Referências Usadas}

[1] Rorato, A.C., Dal’Asta, A.P., Lana, R.M. et al. Trajetorias: a dataset of environmental, epidemiological, and economic indicators for the Brazilian Amazon. Sci Data 10, 65 (2023). https://doi.org/10.1038/s41597-023-01962-1.

% \noindent [2] Prasad, R., Sagar, S.K., Parveen, S. et al. Mathematical modeling in perspective of vector-borne viral infections: a review. Beni-Suef Univ J Basic Appl Sci 11, 102 (2022). https://doi.org/10.1186/s43088-022-00282-4.

\noindent [3] Vyhmeister E., Provan G., Doyle B., Bourke B. Multi-cluster and environmental dependant vector born disease models. Heliyon, Volume 6, Issue 9 (2020). https://doi.org/10.1016/j.heliyon.2020.e04090

\noindent [4] Arquam, M., Singh, A., Cherifi, H. (2020). Integrating Environmental Temperature Conditions into the SIR Model for Vector-Borne Diseases. In: Cherifi, H., Gaito, S., Mendes, J., Moro, E., Rocha, L. (eds) Complex Networks and Their Applications VIII. COMPLEX NETWORKS 2019. Studies in Computational Intelligence, vol 881. Springer, Cham. https://doi.org/10.1007/978-3-030-36687-2\_34.

\noindent [5] ALVES, Leon Diniz. Weather-driven mathematical models of dengue transmission dynamics in twelve Brazilian sites. 2021. 137 f. Tese (Doutorado em Biologia Computacional e Sistemas) - Instituto Oswaldo Cruz, Fundação Oswaldo Cruz, Rio de Janeiro, 2021. https://www.arca.fiocruz.br/handle/icict/52536	.

\noindent [6] Alves, Leon Diniz, Raquel Martins Lana, and Flávio Codeço Coelho. 2021. ``A Framework for Weather-Driven Dengue Virus Transmission Dynamics in Different Brazilian Regions" International Journal of Environmental Research and Public Health 18, no. 18: 9493. https://doi.org/10.3390/ijerph18189493.

% \noindent [7] Prasad, R., Sagar, S.K., Parveen, S. et al. Mathematical modeling in perspective of vector-borne viral infections: a review. Beni-Suef Univ J Basic Appl Sci 11, 102 (2022). https://doi.org/10.1186/s43088-022-00282-4.

% \noindent [8] Abdullah, Seadawy, A. \& Jun, W. New mathematical model of vertical transmission and cure of vector-borne diseases and its numerical simulation. Adv Differ Equ 2018, 66 (2018). 
% https://doi.org/10.1186/s13662-018-1516-z.

% \noindent [9] N. Shah and J. Gupta, "SEIR Model and Simulation for Vector Borne Diseases," Applied Mathematics, Vol. 4 No. 8A, 2013, pp. 13-17. doi: 10.4236/am.2013.48A003.

\end{document}