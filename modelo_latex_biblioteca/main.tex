% -----------------------------------
% -----------------------------------
% abnTeX2: Normas ABNT NBR 14724:2011 + sugestões FGV/EMAp. 

% Autor: Lauro César Araujo
% Adaptações EMAp: Lucas Machado Moschen 
% Copyright 2012-2018 by abnTeX2 group at http://www.abntex.net.br/ 

%% This work may be distributed and/or modified under the
%% conditions of the LaTeX Project Public License, either version 1.3
%% of this license or (at your option) any later version.
%% The latest version of this license is in
%%   http://www.latex-project.org/lppl.txt
%% and version 1.3 or later is part of all distributions of LaTeX
%% version 2005/12/01 or later.
% ----------------------------------
% ----------------------------------
\documentclass[
	% -- opções da classe memoir --
	12pt,				% tamanho da fonte
	%openright,			% capítulos começam em página ímpar (insere página vazia caso preciso)
	oneside,			% para impressão em recto e verso. Oposto a oneside
	a4paper,			% tamanho do papel. 
	% -- opções da classe abntex2 --
	%chapter=TITLE,		% títulos de capítulos convertidos em letras maiúsculas
	%section=TITLE,		% títulos de seções convertidos em letras maiúsculas
	%subsection=TITLE,	% títulos de subseções convertidos em letras maiúsculas
	%subsubsection=TITLE,% títulos de subsubseções convertidos em letras maiúsculas
	% -- opções do pacote babel --
	english,			% idioma para inglês
	brazil				% idioma para português
	]{abntex2}

%------------------------------------------------
%-------------- Pacotes necessários -------------
%------------------------------------------------

% Escrita 
\usepackage[T1]{fontenc}
\usepackage[utf8]{inputenc}
\usepackage{lmodern}
\usepackage{microtype} % para melhorias de justificação
\usepackage{indentfirst}

\renewcommand{\ABNTEXchapterfont}{\fontfamily{ptm}\fontseries{b}\selectfont}

% Gráficos 
\usepackage{color}
\usepackage{caption}
\usepackage{subcaption}
\usepackage{graphicx}
\graphicspath{{../../images/}}

% Matemáticos 
\usepackage{amsthm, amssymb, amsmath, mathtools}

% Outros 
\usepackage{lipsum}


% Citações 
%\usepackage[brazilian,hyperpageref]{backref}
%\usepackage[alf]{abntex2cite}	% Citações padrão ABNT
\usepackage[style=abnt]{biblatex}
\addbibresource{biblio.bib}  

% \renewcommand{\backrefpagesname}{Citado na(s) página(s):~}
% % Texto padrão antes do número das páginas
% \renewcommand{\backref}{}
% % Define os textos da citação
% \renewcommand*{\backrefalt}[4]{
% 	\ifcase #1 %
% 		Nenhuma citação no texto.%
% 	\or
% 		Citado na página #2.%
% 	\else
% 		Citado #1 vezes nas páginas #2.%
% 	\fi}%
% ---

%----------------------------------------
%------- Capa e Folha de Rosto ----------
%----------------------------------------

\newcommand\subtitulo[1]{\def\@subtitulo{#1}}
\newcommand{\imprimirsubtitulo}%{\@subtitulo}

\renewcommand{\imprimircapa}{%
	\begin{capa}%
	\center
		\ABNTEXchapterfont\Large \MakeUppercase{\imprimirinstituicao}
		\\\vspace*{4cm}
		{\ABNTEXchapterfont\large \MakeUppercase{\imprimirautor}}
		\vfill
		\begin{center}
		\ABNTEXchapterfont\large\MakeUppercase{\imprimirtitulo}%\normalfont\MakeUppercase{:
		%\imprimirsubtitulo}
		\end{center}
		\vfill
		\normalfont\large\imprimirlocal
		\\\normalfont\large\imprimirdata
		\vspace*{1cm}
	\end{capa}
}

\makeatletter
\renewcommand{\folhaderostocontent}{
  \begin{center}

    %\vspace*{1cm}
    {\ABNTEXchapterfont\large\MakeUppercase{\imprimirautor}}
	
    \vspace*{\fill}\vspace*{\fill}
    \begin{center}
      \ABNTEXchapterfont\bfseries\large\MakeUppercase{\imprimirtitulo}
      % \ABNTEXchapterfont\bfseries\large\MakeUppercase{\imprimirtitulo}\normalfont\MakeUppercase{:
      % \imprimirsubtitulo}
    \end{center}
    \vspace*{\fill}
	
    \abntex@ifnotempty{\imprimirpreambulo}{%
      \hspace{7.5cm}
      \begin{minipage}{.5\textwidth}
      	\SingleSpacing
         \imprimirpreambulo
         \\\\
         Orientador: \imprimirorientador
       \end{minipage}%
       \vspace*{\fill}
    }%

    % {\large\imprimirorientadorRotulo~\imprimirorientador\par}
    % \abntex@ifnotempty{\imprimircoorientador}{%
    %    {\large\imprimircoorientadorRotulo~\imprimircoorientador}%
    % }%
    \vspace*{\fill}

    {\large\imprimirlocal}
    \par
    {\large\imprimirdata}
    \vspace*{1cm}

  \end{center}
}
\makeatother

\titulo{Utilização de indicadores ambientais e epidemiológicos no estudo da dinâmica da malária}
\autor{Raphael Felberg Levy}
\local{Rio de Janeiro}
\data{2023}
\instituicao{%
  Fundação Getulio Vargas \\
  \par
  Escola de Matemática Aplicada
}
\tipotrabalho{Trabalho de Conclusão de Curso}

\preambulo{Trabalho de conclusão de curso apresentada para a Escola de
Matemática Aplicada (FGV/EMAp) como requisito para o grau de bacharel em
Matemática Aplicada. \\ \\ Área de estudo: Modelagem biológica.}

\orientador{Flávio Codeço Coelho}

% Se o seu texto tem subtítulo. 
% Se não tiver, altere o arquivo capa_folha_rosto_tex
% \subtitulo{Este é o subtítulo do meu TCC}

%---------------------------------------------
%-------------------- PDF --------------------
%---------------------------------------------

% alterando o aspecto da cor azul
\definecolor{blue}{RGB}{41,5,195}

% informações do PDF
\makeatletter
\hypersetup{
     	%pagebackref=true,
		pdftitle={\@title}, 
		pdfauthor={\@author},
    	pdfsubject={\imprimirpreambulo},
	    pdfcreator={LaTeX with abnTeX2},
		pdfkeywords={abnt}{latex}{abntex}{abntex2}{trabalho acadêmico}, 
		colorlinks=true,       		% false: boxed links; true: colored links
    	linkcolor=blue,          	% color of internal links
    	citecolor=blue,        		% color of links to bibliography
    	filecolor=magenta,      		% color of file links
		urlcolor=blue,
		bookmarksdepth=4
}
\makeatother

% Posiciona figuras e tabelas no topo da página quando adicionadas sozinhas
% em um página em branco. Ver https://github.com/abntex/abntex2/issues/170
\makeatletter
\setlength{\@fptop}{5pt} % Set distance from top of page to first float
\makeatother

%---------------------------------------
%--------- Mais configurações-----------
%---------------------------------------

% Possibilita criação de Quadros e Lista de quadros.
% Ver https://github.com/abntex/abntex2/issues/176
\newcommand{\quadroname}{Quadro}
\newcommand{\listofquadrosname}{Lista de quadros}

\newfloat[chapter]{quadro}{loq}{\quadroname}
\newlistof{listofquadros}{loq}{\listofquadrosname}
\newlistentry{quadro}{loq}{0}

% configurações para atender às regras da ABNT
\setfloatadjustment{quadro}{\centering}
\counterwithout{quadro}{chapter}
\renewcommand{\cftquadroname}{\quadroname\space} 
\renewcommand*{\cftquadroaftersnum}{\hfill--\hfill}

\setfloatlocations{quadro}{hbtp} % Ver https://github.com/abntex/abntex2/issues/176

%-----------------------------------------------------
%--------------------- Margens -----------------------
%-----------------------------------------------------

\setlrmarginsandblock{3cm}{2cm}{*}
\setulmarginsandblock{3cm}{2cm}{*}
\checkandfixthelayout

%-----------------------------------------------------
%------ Espaçamentos entre linhas e parágrafos -------
%-----------------------------------------------------

% O tamanho do parágrafo é dado por:
\setlength{\parindent}{1.3cm}

% Controle do espaçamento entre um parágrafo e outro:
\setlength{\parskip}{0.2cm}  % tente também \onelineskip

% compila o índice
\makeindex

%------------------------------------------------------
%----------- Personal Definitions ---------------------
%------------------------------------------------------

\newcommand{\R}{\mathbb{R}}
\newcommand{\x}{\boldsymbol{x}}
\newcommand{\N}{\operatorname{Normal}}
\newcommand{\betadist}{\operatorname{Beta}}
\newcommand{\bern}{\operatorname{Bernoulli}}
\newcommand{\tril}{\operatorname{tril}}

\newcommand{\ev}{\mathbb{E}}
\newcommand{\var}{\operatorname{Var}}
\newcommand{\cor}{\operatorname{Cor}}
\newcommand{\cov}{\operatorname{Cov}}

\newtheorem{theorem}{Theorem}[]
\newtheorem{proposition}{Proposition}[]

\theoremstyle{definition}
\newtheorem{definition}{Definition}[section]

\theoremstyle{remark}
\newtheorem*{remark}{Remark}
\newtheorem{assumption}{Assumption}

\newcommand{\improve}[1]{\textcolor{red}{#1}}

%-------------------------------------------------
%----------------- Document ----------------------
%-------------------------------------------------

\begin{document}

\newcounter{num}
% if num != 1, do not print the pre textual 
\setcounter{num}{1}

\selectlanguage{brazil}
\frenchspacing 

%----------------------------------------------
%--------------- Pré-textuais -----------------
%----------------------------------------------
%\pretextual

\imprimircapa

\ifnum\value{num}=1
{\imprimirfolhaderosto*

\begin{fichacatalografica}
	\sffamily
	\vspace*{\fill}					% Posição vertical
	\begin{center}					
	\fbox{\begin{minipage}[c][8cm]{13.5cm}		% Largura
	\small
	Ficha catalográfica elaborada pela BMHS/FGV \\

	%\imprimirautor
	Felberg Levy, Raphael % Paginas com as citações na bibl
	
	\hspace{0.5cm} \imprimirtitulo  / \imprimirautor. -- \imprimirdata.
	% \hspace{0.5cm} \imprimirtitulo: \imprimirsubtitulo  / \imprimirautor. -- \imprimirdata.


	\hspace{0.5cm} \thelastpage f.\\
		
	\hspace{0.5cm}
	\parbox[t]{\textwidth}{\imprimirtipotrabalho~--~Escola de Matemática Aplicada.}\\
	
	\hspace{0.5cm} Advisor: \imprimirorientador .

	\hspace{0.5cm} Includes bibliography. \\
	
	\hspace{0.5cm}
		1. Matemática
		2. Aplicada
		2. na matemática
		I. Codeço Coelho, Flávio
		II. Escola de Matemática Aplicada
		III. \imprimirtitulo 			
	\end{minipage}}
	\end{center}
\end{fichacatalografica}

% Uncomment if you have the pdf 
% \begin{fichacatalografica}
%     \includepdf{fig_ficha_catalografica.pdf}
% \end{fichacatalografica}

%\begin{errata}

    \begin{table}[htb]
        \center
        \footnotesize
        \begin{tabular}{|p{1.4cm}|p{1cm}|p{3cm}|p{3cm}|}
        \hline
        \textbf{Folha} & \textbf{Linha} & \textbf{Onde se lê} &
        \textbf{Leia-se}\\
        \hline
        17 & 8 & Matemtica & Matemática \\
        \hline
        \end{tabular}
    \end{table}
    
    \end{errata}

\begin{folhadeaprovacao}

    \begin{center}
      {\ABNTEXchapterfont\large\MakeUppercase{\imprimirautor}}
  
      \vspace*{\fill}\vspace*{\fill}
      \begin{center}
        \ABNTEXchapterfont\bfseries\large\MakeUppercase{\imprimirtitulo}%\normalfont\MakeUppercase
        %{:\imprimirsubtitulo}	
      \end{center}
      \vspace*{\fill}
      
      \hfill
      \begin{minipage}{.7\textwidth}
          \imprimirpreambulo \\ \\
          E aprovado em 12/12/2023 \\
          Pela comissão organizadora
      \end{minipage}%
      \vspace*{\fill}
     \end{center}
  
     \assinatura{\imprimirorientador \\ Escola de Matemática Aplicada} 
     \assinatura{Claudio José Struchiner \\ Escola de Matemática Aplicada}
     \assinatura{Mônica da Silva-Nunes \\ Universidade Federal de São Carlos - UFSCar}
     %\assinatura{\textbf{Professor} \\ Convidado 3}
     %\assinatura{\textbf{Professor} \\ Convidado 4}
\end{folhadeaprovacao}

% \begin{folhadeaprovacao}
% \includepdf{folhadeaprovacao_final.pdf}
% \end{folhadeaprovacao}

% \begin{dedicatoria}
%     \vspace*{\fill}
%     %\noindent
%     \hfill
%     \begin{minipage}{.6\textwidth}
%      Dedico essa dissertação a todas que lutaram para que eu estivesse aqui. 
%     \end{minipage}
% \end{dedicatoria}
 
\begin{agradecimentos}
    À minha família, especialmente meus pais, por todo o apoio e 
    incentivo ao longo não só da graduação, como em toda a jornada até esse momento. 
    \\\\
    Ao meu orientador, Flávio Codeço Coelho, por ser meu guia no
    desenvolvimento desse Trabalho e por me apresentar à área da modelagem de fenômenos biológicos.   
    \\\\
    A todos os professores que tive a oportunidade de conhecer e com quem tive o prazer de aprender 
    ao longo da graduação, e aos monitores que se dispunham a ajudar nos momentos mais difíceis.
    \\\\
    E por fim, gostaria de agradecer a todos os meus amigos que me acompanharam e me apoiaram até aqui. 
    Os últimos 4 anos não seriam os mesmos sem vocês.
\end{agradecimentos}

% \begin{epigrafe}
% \vspace*{\fill}

% \begin{flushright}
%     \hspace{7.5cm}
%     \textit{
%         ``If your experiment needs a statistician, you need a better
%         experiment.''} \\
%         \textit{Ernest Rutherford}
% \end{flushright}
% \end{epigrafe}

\setlength{\absparsep}{18pt} 
\begin{resumo}[Resumo]
    A malária é uma doença infecciosa transmitida por mosquitos infectados por protozoários do gênero \textit{Plasmodium}, sendo a região amazônica considerada área endêmica 
    para a doença. Esse trabalho tem como intuito analisar o comportamento dessa transmissão baseado em modificações climáticas e ambientais, como temperatura, precipitação e desmatamento, 
    através de modificações propostas aos modelos SIR e SEI, de forma a contribuir no estudo de aplicações 
    de efeitos externos na evolução da doença. 
    O Projeto Trajetórias, desenvolvido pelo Centro de Biodiversidade e Serviços Ecossistêmicos (SinBiose/CNPq), será usado como base de referência para as análises.
    

 Palavras-chave: Modelagem biológica. Malária. Amazônia. SIR. SEI.
\end{resumo}

\begin{resumo}[Abstract]
 \begin{otherlanguage*}{english}
    Malaria is an infectious disease transmitted by mosquitoes infected by protozoa of the genus \textit{Plasmodium}, with the Amazon region being considered an endemic area 
    for the disease. This work aims to analyze the behavior of this transmission based on climatic and environmental changes, such as temperature, precipitation and deforestation, through proposed 
    modifications to the SIR and SEI models, in order to contribute to the study of applications
    of external effects on the evolution of the disease. 
    The Trajetórias Project, developed by the Synthesis Center 
    on Biodiversity and Ecosystem Services (SinBiose/CNPq) will be used as a reference base for the analyses.
    
 \end{otherlanguage*}

 Keywords: Biological modelling. Malaria. Amazon. SIR. SEI.
\end{resumo}

\pdfbookmark[0]{\listfigurename}{lof}
\listoffigures*
\cleardoublepage

% \pdfbookmark[0]{\listofquadrosname}{loq}
% \listofquadros*
% \cleardoublepage

\pdfbookmark[0]{\listtablename}{lot}
\listoftables*
\cleardoublepage

% \begin{siglas}
%     \item[ABNT] Associação Brasileira de Normas Técnicas
%     \item[abnTeX] ABsurdas Normas para TeX
%   \end{siglas}
  
%   \begin{simbolos}
%     \item[$ \Gamma $] Letra grega Gama
%     \item[$ \Lambda $] Lambda
%     \item[$ \zeta $] Letra grega minúscula zeta
%     \item[$ \in $] Pertence
%   \end{simbolos}

}\fi

\pdfbookmark[0]{\contentsname}{toc}
\tableofcontents*
\cleardoublepage

% ----------------------------------------------------------
% ELEMENTOS TEXTUAIS
% ----------------------------------------------------------
\textual

\chapter{Introdução}

A Amazônia é uma das maiores e mais biodiversas florestas tropicais do mundo, 
abrigando inúmeras espécies de plantas, animais e microrganismos, incluindo 
vetores e patógenos responsáveis pela transmissão de diversas doenças. Entre 
elas, uma das mais comuns é a malária, que é causada por protozoários do 
gênero \textit{Plasmodium}, transmitidos pela picada da fêmea infectada do 
mosquito do gênero \textit{Anopheles}. Ela está presente em 22 países 
americanos, porém as áreas com maior risco de infecção estão localizadas 
na região amazônica, englobando nove países, e que representaram $68\%$ 
dos casos de infecção em 2011 \cite{pimenta_orfano_bahia_duarte_rios-velasquez_melo_pessoa_oliveira_campos_villegas_etal_2015}. Apesar de ser muito comum nas 
Américas, a malária não é limitada a esse continente, sendo encontrada 
em países da África e Ásia, tendo resultado em mais de dois milhões de 
casos de infecção e  445 mil mortes ao redor do mundo em 2016 \cite{doi:10.1146/annurev-micro-090817-062712}.    
\\\\
Notavelmente, a transmissão de doenças por vetores é intimamente relacionada 
a alterações ambientais que interferem no ecossistema dos organismos 
transmissores e dos organismos afetados. No caso da Amazônia, povoados 
agrícolas e agropecuários são alguns dos fatores que mais favorecem a 
transmissão da doença, tanto pelo desmatamento que causam para seu 
estabelecimento, quanto pelo agrupamento de pessoas em ambientes 
próximos ao habitat do vetor \cite{silva-nunes_malaria_amazon_2008}, em especial por aglomerar migrantes 
não-imunes próximos a esses criadouros naturais e artificiais \cite{DASILVANUNES2012281}. 
\\\\
Além disso, outros fatores, 
como chuvas, queimadas e mineração também são muito influentes na 
transmissão de doenças na região. Esses eventos resultam em perda 
de habitat, fragmentação de ecossistemas e alterações no clima, 
afetando a distribuição e abundância de vetores e hospedeiros, bem 
como a interação entre eles e os patógenos. Ademais, o crescimento 
populacional e a urbanização também têm um papel importante na disseminação 
de doenças, uma vez que aumentam a exposição dos seres humanos aos vetores 
e aos riscos de infecção.
\\\\
Diante desse contexto, este trabalho visa investigar a transmissão de 
doenças por vetores na Amazônia e analisar como os impactos ambientais 
influenciam a dinâmica de transmissão da malária, os fatores ecológicos 
e socioeconômicos que afetam essa disseminação e possíveis estratégias 
de prevenção e controle, tendo como referência principal o Projeto 
Trajetórias, desenvolvido pelo Centro de Biodiversidade e Serviços Ecossistêmicos (SinBiose/CNPq), que é um dataset incluindo 
indicadores ambientais, epidemiológicos, econômicos e socioeconômicos 
para todos os municípios da Amazônia Legal, analisando a relação espacial 
e temporal entre trajetórias econômicas ligadas à dinâmica dos sistemas 
agrários, sendo eles rurais de base familiar ou produção agrícola e de 
gado em larga escala, a disponibilidade de recursos naturais e o risco 
de doenças \cite{Rorato2023}.

% ----------------------------------------------------------
% Finaliza a parte no bookmark do PDF
% para que se inicie o bookmark na raiz
% e adiciona espaço de parte no Sumário
% ----------------------------------------------------------
\phantompart

\chapter{Metodologia}

Corresponde ao corpo do trabalho, contendo a exposição ordenada e pormemorizada
do assunto. Constam aqui a revisão de literatura, metodologia adotada, os resultados e
sua discussão. Divide-se em seções e subseções. \cite{pimenta_orfano_bahia_duarte_rios-velasquez_melo_pessoa_oliveira_campos_villegas_etal_2015}

Para a elaboração do trabalho, serão usados dados populacionais 
do dataset do Projeto Trajetórias e dados climáticos do Climate Data, 
e serão abordados métodos de transmissão de doenças baseados em equações 
diferenciais ordinárias, como o SIR, e, partindo de uma modelagem simples, 
serão incluídos os fenômenos ambientais, como desmatamento e queimada, para 
verificar como modificações no ecossistema irão interferir no modelo elaborado 
previamente. Os cálculos computacionais foram realizados em ambiente SageMath 9.2, 
utilizando funções de integração numérica do Scipy para solução do método.
\\\\
Descrevendo primeiramente SIR $^{[6], [7]}$, que pode ser considerado a base de modelos que serão usados ao longo do projeto, este foi desenvolvido por W. O. Kermack e A. G. McKendrick em 1927, sendo um dos modelos mais usados para a modelagem de epidemias, levando em consideração três compartimentos:

\begin{align*}
    & S: \text{número de indivíduos suscetíveis} \\
    & I: \text{número de indivíduos infectados} \\
    & R: \text{número de indivíduos recuperados}
\end{align*}
\\
Nesse modelo, os indivíduos saudáveis na classe $S$ são suscetíveis ao contato com indivíduos da classe $I$, e são transferidos para esse compartimento caso contraiam a doença. Indivíduos infectados podem espalhar a doença por contato direto com indivíduos suscetíveis, mas também podem se tornar imunes ao longo do tempo, sendo transferidos para o compartimento $R$. Em geral, $R$ inclui o total de recuperados (imunes) e mortos em decorrência da doença, mas podemos assumir que o número de mortos é muito baixo em relação ao tamanho da população total, podendo ser ignorado. Consideramos também que indivíduos nessa categoria não voltarão a ser suscetíveis ou infecciosos.   
\\\\
Considerando uma epidemia em um espaço curto de tempo e que a doença não é fatal, podemos ignorar dinâmicas vitais de nascimento e morte. Com isso, podemos descrever o modelo SIR através do seguinte sistema de EDOs:

\begin{gather*}
\begin{cases}
\dfrac{dS}{dt} = -\dfrac{\beta SI}{N} \\
\\
\dfrac{dI}{dt} = \dfrac{\beta SI}{N} - \gamma I \\
\\
\dfrac{dR}{dt} = \gamma I
\end{cases}
\end{gather*}
\\
No modelo, $N(t) = S(t)+I(t)+R(t)$, ou seja, a população total no tempo $t$, enquanto que $\beta$ é a taxa de infecção e $\gamma$ é a taxa de recuperação. Dado que $S+I+R$ é sempre constante se ignorarmos nascimento e morte, temos $\dfrac{dS}{dt}+\dfrac{dI}{dt}+\dfrac{dR}{dt} = 0$. 
\\\\
Para que a doença possa se espalhar, é fácil ver que $\dfrac{dI}{dt} = \dfrac{\beta SI}{N} - \gamma I > 0$. Assim, $\dfrac{\beta SI}{N} > \gamma I \Rightarrow \dfrac{\beta S}{N} > \gamma$. Supondo que estamos no início da infeccção, dado que queremos ver como se espalha, $I$ será muito pequeno e $S \approx N$. Concluímos então que $\dfrac{\beta N}{N} > \gamma \Rightarrow \dfrac{\beta}{\gamma} > 1$. É possível derivar esse valor adimensionalizando o modelo: sejam $y^* = \dfrac{S}{N}, \ x^* = \dfrac{I}{N}, \ z^* = \dfrac{R}{N}$ e $t^*=\dfrac{t}{1/\gamma} = \gamma t$, de forma que $y^*+x^*+z^*=1$. Substituindo o sistema de EDOs acima utilizando esses valores:

\begin{gather*}
\begin{cases}
\dfrac{dS}{dt} = \dfrac{d(y^*N)}{d(t^*/\gamma)} = -\dfrac{\beta SI}{N} = -\dfrac{\beta(y^*N)(x^*N)}{N} = -\beta y^*Nx^* \\
\\
\dfrac{dI}{dt} = \dfrac{d(x^*N)}{d(t^*/\gamma)} = \dfrac{\beta SI}{N} - \gamma I = \dfrac{\beta(y^*N)(x^*N)}{N} -\gamma(x^*N) = \beta y^*Nx^* - \gamma x^*N \\
\\
\dfrac{dR}{dt} = \dfrac{d(z^*N)}{d(t^*/\gamma)} = \gamma I = \gamma(x^*N)
\end{cases}
\end{gather*}

Agora, cancelando os fatores $N$ e $\gamma$ em ambos os lados das equações:

\begin{gather*}
\begin{cases}
\dfrac{d(y^*)}{d(t^*)} = -\dfrac{\beta y^*x^*}{\gamma} \\
\\
\dfrac{d(x^*)}{d(t^*)} = \dfrac{\beta y^*x^*}{\gamma} - x^* \\
\\
\dfrac{d(z^*)}{d(t^*)} = x^*
\end{cases}
\end{gather*}
\\
Sendo assim temos um sistema dado apenas por $y^*$ e $x^*$ e o parâmetro 
$\dfrac{\beta}{\gamma}$, que podemos chamar de $R_0$.
\\\\
Como esse trabalho será focado principalmente na modelagem de malária, 
irei agora apresentar um dos primeiros modelos desenvolvidos especialmente 
para essa doença, por Sir Ronald Ross em 1911 $^{[8]}$, que usa duas EDOs 
distintas das apresentadas acima:

\begin{gather*}
\begin{cases}
\dfrac{dI}{dt} = bp'i\dfrac{N-I}{N} -aI\\
\\
\dfrac{di}{dt} = bp(n-i)\dfrac{I}{N} - mI
\end{cases}
\end{gather*}
\\
Nesse caso, $N$ é a população humana total, $I(t)$ é o número de humanos 
infectados no tempo $t$, $n$ é a população total de mosquitos, $i(t)$ é o 
número de mosquitos infectados no tempo $t$, $b$ é a taxa de picadas, $p$ 
é a probabilidade de transmissão do humano para o mosquito por picada, $p'$ 
é a probabilidade de transmissão do mosquito para o humano por picada, $a$ é 
a taxa de recuperação da infecção de um humano e $m$ é a taxa de mortalidade 
dos mosquitos. $bp'i\dfrac{N-I}{N}dt -aIdt$ representam respectivamente o 
número de novos humanos infectados e o número de humanos recuperados no 
intervalo $dt$, enquanto que $bp(n-i)\dfrac{I}{N}dt - mIdt$ representam 
respectivamente o número de novos mosquitos infectados e o número de 
mosquitos que morrem nesse intervalo de tempo, assumindo que a infecção 
não interfere na taxa de mortalidade dos mosquitos.
\\\\
Para esse modelo, Ross discutiu dois pontos de equilíbrio, em que 
$\dfrac{dI}{dt} = \dfrac{di}{dt} = 0$. Eles ocorrem quando $I=i=0$, 
que é o caso onde não existe malária, e, para $I, i > 0$, 
$I = N\dfrac{1-amN/(b^2pp'n)}{1+aN/(bp'n)}$ e 
$i = n\dfrac{1-amN/(b^2pp'n)}{1+m/(bp)}$. Ainda, para que a doença se 
estabeleça, $n$ deve ser maior que um valor limiar $n^* = \dfrac{amN}{b^2pp'}$. 
Nesse caso a doença se torna endêmica. Caso $n<n^*$, o equilíbrio estará em 
$I=i=0$ e a dença irá desaparecer.
\\\\
Dividindo as equações dos pontos de equilíbrio por $I \times i$, temos:

\begin{gather*}
\begin{cases}
\dfrac{bp}{N} = \dfrac{bpn}{Ni} -\dfrac{m}{I} \\
\\
\dfrac{bp'}{N} = \dfrac{bp'}{I} -\dfrac{a}{i} 
\end{cases}
\end{gather*}
\\
O que transforma o problema em um sistema linear com dois desconhecidos, 
$I$ e $i$.
\\\\
Agora, irei apresentar o modelo que será usado para o desenvolvimento do 
trabalho, feito com base no elaborado por Paul E. Parham e Edwin Michael 
em 2010, que leva em consideração fatores como a chuva e temperatura 
($R$ e $T$, respectivamente) $^{[9]}$. 
\\\\
Definindo as equações que serão utilizadas:
\begin{gather*}
\begin{cases}
\dfrac{dS_H}{dt} = -ab_2\bigg(\dfrac{I_M}{N}\bigg)S_H\\
\\
\dfrac{dI_H}{dt} = ab_2\bigg(\dfrac{I_M}{N}\bigg)S_H-\gamma I_H\\
\\
\dfrac{dR_H}{dt} = \gamma I_H\\
\\
\dfrac{dS_M}{dt} = b - ab_1\bigg(\dfrac{I_H}{N}\bigg)S_M - \mu S_M\\
\\
\dfrac{dE_M}{dt} = ab_1\bigg(\dfrac{I_H}{N}\bigg)S_M - \mu E_M - ab_1\bigg(\dfrac{I_H}{N}\bigg)S_Ml(\tau_M)\\
\\
\dfrac{dI_M}{dt} = ab_1\bigg(\dfrac{I_H}{N}\bigg)S_Ml(\tau_M) -\mu I_M
\end{cases}
\end{gather*}
\\\\
É preciso comentar que o modelo original utilizava $I_M(t-\tau)$ em 
$\dfrac{dI_H}{dt}$ e $I_H(t-\tau)$ em $\dfrac{dE_M}{dt}$ (na passagem 
de $E$ para $I$) e $\dfrac{dI_M}{dt}$, respectivamente, mas como isso 
faria com que o modelo fosse baseado em equações com atraso, foi 
recomendado pelo orientador do Trabalho que essa diferença fosse 
desconsiderada, e usasse apenas $t$ atual.
\\\\
Tendo as equações do modelo para a população de humanos e de mosquitos, 
irei primeiro definir os parâmetros utilizados na modelagem e outras 
funções necessárias, e depois as variáveis usadas:
\\
\begin{adjustwidth}{-0.5cm}{}
\begin{center}
\renewcommand{\arraystretch}{1.5}
\raggedleft\begin{tabular}{|c | l | c|} 
 \hline
 \raisebox{-1ex}{\textbf{Parâmetro}} & \raisebox{-1ex}{\textbf{Definição}} & \raisebox{-1ex}{\textbf{Cálculo}}\\ 
 \hline
 $T(t)$ & \pbox{8cm}{\rule{0pt}{4.5ex}Temperatura\rule[-2.5ex]{0pt}{0pt}} & $T_1 (1 + T_2 \cos(\omega_1t - \phi_1))$\\ 
 \hline
 $R(t)$ & \pbox{8cm}{\rule{0pt}{4.5ex}Precipitação\rule[-2.5ex]{0pt}{0pt}} & $R_1 (1 + R_2 \cos(\omega_2t - \phi_2))$ \\
 \hline
 $b(R, T)$ & \pbox{8cm}{\rule{0pt}{4.5ex}Taxa de nascimento de mosquitos (/ dia)\rule[-2.5ex]{0pt}{0pt}} & $\dfrac{B_E  p_E(R)  p_L(R,T)  p_P(R)}{(\tau_E + \tau_L(T) + \tau_P)}$\\ 
 \hline
 $a(T)$ & \pbox{8cm}{\rule{0pt}{4.5ex}Taxa de picadas (/dia)\rule[-2.5ex]{0pt}{0pt}} & $\dfrac{(T - T_1)}{D_1}$ \\
 \hline
 $\mu(T)$ & \pbox{8cm}{\rule{0pt}{3ex}Taxa de mortalidade de mosquitos per capita (/ dia)\rule[-1.5ex]{0pt}{0pt}} & $-\log(p(T))$ \\
 \hline
 $\tau_M(T)$ & \pbox{8cm}{\rule{0pt}{4.5ex}Duração do ciclo de esporozoitos (dias)\rule[-2.5ex]{0pt}{0pt}} & $\dfrac{DD}{(T - T_{min})}$ \\
 \hline
 $\tau_L(T)$ & \pbox{8cm}{\rule{0pt}{4.5ex}Duração da fase de desenvolvimento das larvas (dias)\rule[-2.5ex]{0pt}{0pt}} & $\dfrac{1}{c_1T + c_2}$ \\
 \hline
 $p(T)$ & \pbox{8cm}{\rule{0pt}{3ex}Taxa diária de sobrevivência dos mosquitos \rule[-1.5ex]{0pt}{0pt}} & $e^{(-1 / (AT^2 + BT + C))}$ \\
 \hline
 $p_L(R)$ & \pbox{8cm}{\rule{0pt}{3ex}Probabilidade de sobrevivência das larvas dependente de chuva\rule[-1.5ex]{0pt}{0pt}} & $(\dfrac{4p_{ML}}{R_L^2})R(R_L - R)$ \\
 \hline
 $p_L(T)$ & \pbox{8cm}{\rule{0pt}{3ex}Probabilidade de sobrevivência das larvas dependente de temperatura\rule[-1.5ex]{0pt}{0pt}} & $e^{-(c_1T + c_2)}$ \\
 \hline
 $p_L(R, T)$ & \pbox{8cm}{\rule{0pt}{3ex}Probabilidade de sobrevivência das larvas dependente de temperatura e chuva\rule[-1.5ex]{0pt}{0pt}} & $p_L(R)p_L(T)$ \\
 \hline
 $l(\tau_M)(T)$ & \pbox{8cm}{\rule{0pt}{3ex}Probabilidade de sobrevivência de mosquitos durante o ciclo de esporozoitos (/ dia)\rule[-1.5ex]{0pt}{0pt}} & $p(T)^{\tau_M(T)}$ \\
 \hline
 $M(t)$ & \pbox{8cm}{\rule{0pt}{3ex}Número total de mosquitos\rule[-1.5ex]{0pt}{0pt}} & $S_M(t) + E_M(t) + I_M(t)$ \\
 \hline
 $N(t)$ & \pbox{8cm}{\rule{0pt}{3ex}Número total de humanos\rule[-1.5ex]{0pt}{0pt}} & $S_H(t) + I_H(t) + R_H(t)$ \\  
 \hline
\end{tabular}
\captionof{table}{Parâmetros usados na modelagem}
\end{center}
\end{adjustwidth}

\begin{adjustwidth}{-0.5cm}{}
\begin{center}
\renewcommand{\arraystretch}{1.5}
\raggedleft\begin{tabular}{|c | c|} 
 \hline
 \textbf{Parâmetro} & \textbf{Definição}\\ 
 \hline
 $b_1$ & \makecell[l]{\rule{0pt}{3ex}Proporção de picadas de mosquitos suscetíveis \\ em humanos infectados que produzem infecção\rule[-1.5ex]{0pt}{0pt}} \\
 \hline
 $b_2$ & \makecell[l]{\rule{0pt}{3ex}Proporção de picadas de mosquitos infectados \\ em humanos suscetíveis que produzem infecção\rule[-1.5ex]{0pt}{0pt}} \\
 \hline
 $\gamma$ & \makecell[l]{\rule{0pt}{3ex}1/Duração média da infecciosidade em humanos (dias$^{-1}$)\rule[-1.5ex]{0pt}{0pt}} \\
 \hline
 $T_1$ & \makecell[l]{\rule{0pt}{3ex}Temperatura média na ausência de sazonalidade ($^\circ C$)\rule[-1.5ex]{0pt}{0pt}} \\
 \hline
 $T_2$ & \makecell[l]{\rule{0pt}{3ex}Amplitude da variabilidade sazonal na temperatura\rule[-1.5ex]{0pt}{0pt}} \\
 \hline
 $R_1$ & \makecell[l]{\rule{0pt}{3ex}Precipitação mensal média na ausência de \\ sazonalidade (mm)\rule[-1.5ex]{0pt}{0pt}} \\
 \hline
 $R_2$ & \makecell[l]{\rule{0pt}{3ex}Amplitude da variabilidade sazonal na precipitação\rule[-1.5ex]{0pt}{0pt}} \\
 \hline
 $\omega_1$ & \makecell[l]{\rule{0pt}{3ex}Frequência angular das oscilações sazonais na temperatura (meses$^{-1}$)\rule[-1.5ex]{0pt}{0pt}} \\
 \hline
 $\omega_2$ & \makecell[l]{\rule{0pt}{3ex}Frequência angular das oscilações sazonais na precipitação (meses$^{-1}$)\rule[-1.5ex]{0pt}{0pt}} \\
 \hline
 $\phi_1$ & \makecell[l]{\rule{0pt}{3ex}``Phase lag" da variabilidade da temperatura (defasagem de fase)\rule[-1.5ex]{0pt}{0pt}} \\
 \hline
 $\phi_2$ & \makecell[l]{\rule{0pt}{3ex}``Phase lag" da variabilidade da precipitação (defasagem de fase)\rule[-1.5ex]{0pt}{0pt}} \\
 \hline
 $B_E$ & \makecell[l]{\rule{0pt}{3ex}Número de ovos colocados por adulto por oviposição\rule[-1.5ex]{0pt}{0pt}} \\
 \hline
 $p_{ME}$ & \makecell[l]{\rule{0pt}{3ex}Probabilidade máxima de sobrevivência dos ovos\rule[-1.5ex]{0pt}{0pt}} \\
 \hline
 $p_{ML}$ & \makecell[l]{\rule{0pt}{3ex}Probabilidade máxima de sobrevivência das larvas\rule[-1.5ex]{0pt}{0pt}} \\
 \hline
 $p_{MP}$ & \makecell[l]{\rule{0pt}{3ex}Probabilidade máxima de sobrevivência das pupas\rule[-1.5ex]{0pt}{0pt}} \\
 \hline
 $\tau_E$ & \makecell[l]{\rule{0pt}{3ex}Duração da fase de desenvolvimento dos ovos (dias)\rule[-1.5ex]{0pt}{0pt}} \\
 \hline
 $b_3^*$ & \makecell[l]{\rule{0pt}{3ex}Taxa de infecção em mosquitos expostos $(1/\tau_M(T))$\rule[-1.5ex]{0pt}{0pt}} \\
 \hline
\end{tabular}
\captionof{table}{Parâmetros usados na modelagem}
\end{center}
\end{adjustwidth}


\begin{adjustwidth}{-0.5cm}{}
\begin{center}
\renewcommand{\arraystretch}{1.5}
\raggedleft\begin{tabular}{|c | c|} 
 \hline
 \textbf{Parâmetro} & \textbf{Definição}\\ 
 \hline
  $\tau_P$ & \makecell[l]{\rule{0pt}{3ex}Duração da fase de desenvolvimento das pupas (dias)\rule[-1.5ex]{0pt}{0pt}} \\
 \hline
 $R_L$ & \makecell[l]{\rule{0pt}{3ex}Chuva limite até que os sítios de reprodução sejam eliminados, \\ removendo indivíduos de estágio imaturo (mm)\rule[-1.5ex]{0pt}{0pt}} \\
 \hline
 $T_{min}$ & \makecell[l]{\rule{0pt}{3ex}Temperatura mínima, abaixo dessa temperatura não há desenvolvimento \\ do parasita: 14.5 ($^\circ C$)\rule[-1.5ex]{0pt}{0pt}} \\
 \hline
 $DD$ & \makecell[l]{\rule{0pt}{3ex}``Degree-days" para desenvolvimento do parasita. Número de graus em \\ que a temperatura média diária excede a temperatura mínima \\ de desenvolvimento.
 ``Sum of heat" para maturação: 105 $(^\circ C \ \text{dias})$ $^{[10], [15]}$\rule[-1.5ex]{0pt}{0pt}} \\
 \hline
 $A$ & \makecell[l]{\rule{0pt}{3ex}Parâmetro empírico de sensibilidade ($^\circ C^2 \ \text{dias})^{-1}$\rule[-1.5ex]{0pt}{0pt}} \\
 \hline
 $B$ & \makecell[l]{\rule{0pt}{3ex}Parâmetro empírico de sensibilidade ($^\circ C \ \text{dias})^{-1}$\rule[-1.5ex]{0pt}{0pt}} \\
 \hline
 $C$ & \makecell[l]{\rule{0pt}{3ex}Parâmetro empírico de sensibilidade ($\text{dias}^{-1}$)\rule[-1.5ex]{0pt}{0pt}} \\
 \hline
 $D_1$ & \makecell[l]{\rule{0pt}{3ex}Constante: 36.5 ($^\circ C \ \text{dias}$)\rule[-1.5ex]{0pt}{0pt}} \\
 \hline
 $c_1$ & \makecell[l]{\rule{0pt}{3ex}Parâmetro empírico de sensibilidade ($^\circ C \ \text{dias})^{-1}$\rule[-1.5ex]{0pt}{0pt}} \\
 \hline
 $c_2$ & \makecell[l]{\rule{0pt}{3ex}Parâmetro empírico de sensibilidade ($\text{dias}^{-1}$)\rule[-1.5ex]{0pt}{0pt}} \\
 \hline
 $T'^*$ & \makecell[l]{\rule{0pt}{3ex}Parâmetro empírico de temperatura ($^\circ C$)\rule[-1.5ex]{0pt}{0pt}} \\
 \hline
\end{tabular}
\captionof{table}{Parâmetros usados na modelagem}
\end{center}
\end{adjustwidth}

\vspace{1cm}
Parâmetros marcados com $^*$ foram adicionados durante o desenvolvimento da modelagem para correção de imprecisões derivadas das equações originais do artigo de referência.
\\\\
Tendo as equações e parâmetros, a modelagem foi feita inicialmente 
utilizando dados da zona rural de Manaus, no período de 2004 a 2008, 
que foram selecionados devido à maior incidência de casos de malária causados 
por $P. \ vivax$, sendo a espécie responsável pelo maior número de
casos no Brasil $^{[11, 12]}$. Usando a função de incidência utilizada no projeto 
Trajetorias $^{[5]}$, temos:
\begin{gather*}
    \text{Inc}(d, m, z, t_1, t_2) = \dfrac{\text{Casos}(d, m, z, t_1, t_2)}{\text{Pop}(m,z,(t_1+t_2)/2) \times 5 \ \text{anos}} \times 10^5,
\end{gather*}
onde $\text{casos}(d, m, z, t_1, t_2)$ é o número de casos da doença $d$ na zona $z$ do município $m$, e $t_1$ e $t_2$ são os anos iniciais e finais do intervalo, enquanto que $\text{pop}(m,z,(t_1+t_2)/2) \times 5 \ \text{anos}$ é a população na zona $z$ do município $m$ no meio do período multiplicado pelo total de anos de observação. Nesse caso, poderíamos indicar como:
\begin{gather*}
    \footnotesize{\text{Inc}(\text{Vivax}, \text{Manaus}, \text{Rural}, 2004, 2008) = \dfrac{\text{Casos}(\text{Vivax}, \text{Manaus}, \text{Rural}, 2004, 2008)}{\text{Pop}(\text{Manaus}, \text{Rural}, 2006) \times 5 \ \text{anos}} \times 10^5}  \\\\
    184030.8 = \dfrac{78745}{5\text{Pop}} \times 10^5 \Rightarrow Pop \approx 8558
\end{gather*}
Usando dados da população total de Manaus nesse período, cuja incidência foi de 3106.4 e o número de casos foi de 262264, a população total do município foi estimada como sendo de 1688540 habitantes. Com isso, a população rural pôde ser considerada como aproximadamente 0.5$\%$ da população do município.
\\\\
Tendo estimado o tamanho porcentual da população rural na cidade, foi possível calcular essa população em cada um dos anos da análise através de uma interpolação linear feita com dados de séries históricas do IBGE $^{[13]}$:
\\\\
\begin{adjustwidth}{0cm}{}
\begin{center}
\renewcommand{\arraystretch}{1.5}
\begin{tabular}{|c | c|} 
 \hline
 \textbf{Ano} & \textbf{População rural estimada}\\ 
 \hline
$2004$ & $7717$ \\
 \hline
 $2005$ & $7889$ \\
 \hline
 $2006$ & $8061$ \\
 \hline
 $2007$ & $8233$ \\
 \hline
 $2008$ & $8492$ \\
 \hline
 $2009$ & $8751$ \\
 \hline
\end{tabular}
\captionof{table}{População rural de Manaus de 2004 a 2009}
\end{center}
\end{adjustwidth}

\vspace{1cm}
Como haviam dados populacionais para os anos de 2000, 2007 e 2010, as interpolações foram feitas com diferentes pontos iniciais e finais, usando de 2000 a 2007 para 2004-2007, e de 2007 a 2010 para 2008-2009, de forma a utilizar corretamente a população de 2007.
\\\\
Descrevendo agora um pouco da teoria por trás dos fatores ambientais, segundo $^{[16]}$, 
a remoção da copa das árvores permitiu a reemergência da malária 
na América do Sul, já que em áreas desmatadas, 
sem as copas cobrindo o solo, poças d'água sob luz solar atraem mosquitos 
da espécie $Anopheles \ darlingi$, principal vetor relacionado à malária 
humana na Amazônia $^{[17]}$, sendo que costumam ser menos encontrados em 
florestas ainda intactas. Isso ocorre porque a luz e calor favorecem o 
desenvolvimento de larvas e pupas, além de uma maior disponibilidade de 
algas para alimentação das larvas $^{[18]}$. O aumento da temperatura 
ambiente também favorece a capacidade vetorial dos mosquitos. O desmatamento 
também atrai e aproxima humanos para que possam tomar parte em atividades 
madeireiras, de agricultura e construção de rodovias, trazendo indíviduos 
infectados com o $Plasmodium$ para uma área em que tanto o vetor quanto o 
ambiente já foram modificados de forma a favorecer a sua transmissão. 
Ainda, a agricultura também favorece a sedimentação dos rios, sendo 
ambientes propícios para o estabelecimento de criadouros. Sendo assim, 
pode ser considerada uma mudança adequada para o modelo para levar 
em consideração o desmatamento, o aumento das probabilidades de 
sobrevivência dos ovos, larvas e pupas, além de aumentar a proporção 
de picadas que produzem infecção, devido ao aumento da densidade 
populacional humana em áreas próximas aos criadouros dos mosquitos.

\chapter{Conclusão}

Ao longo do desenvolvimento do TCC, foram exploradas diferentes modificações às dinâmicas 
de transmissão da malária na Amazônia, de forma a aproximar a modelagem do mais compatível
com a história natural da doença nesse ambiente, com o objetivo final de entender
como impactos ecológicos na região afetam as interações entre vetor e hospedeiro.
\\\\
Com os resultados obtidos, foi possível
perceber o efeito que o maior contato entre humanos e mosquitos devido ao desmatamento
pode ter na dinâmica da malária com base na proporção de picadas causando infecção. 
Mais ainda, foi possível verificar como, dependendo dos parâmetros originais passados,
será necessária uma aproximação muito mais elevada entre vetor e hospedeiro para 
que a doença se torne endêmica na região amazônica. Como verificado, é possível que esse
contato chegue ao dobro do que é normalmente, e isso ainda não é suficiente 
para que a doença se torne uma epidemia.
\\\\
Para aproximar ainda mais os métodos usados aos comportamentos verificados na realidade,
poderia ser ideal a aplicação de um modelo de transmissão estocástico, incorporando as 
variáveis ambientais em constante mudança, mas para o que foi proposto, o modelo determinístico
utilizado foi suficiente para destacar as sensibilidades da doença às alterações 
climáticas e ambientais, permitindo uma análise clara e direcionada 
das interações entre vetor e hospedeiro, fornecendo uma base sólida 
para investigar as implicações das mudanças ambientais na transmissão da malária e para futuras investigações e aprimoramentos nos modelo.


%% \chapter{Apêndices}

% A Amazônia é uma das maiores e mais biodiversas florestas tropicais do mundo, 
% abrigando inúmeras espécies de plantas, animais e microrganismos, incluindo 
% vetores e patógenos responsáveis pela transmissão de diversas doenças. Entre 
% elas, uma das mais comuns é a malária, que é causada por protozoários do 
% gênero \textit{Plasmodium}, transmitidos pela picada da fêmea infectada do 
% mosquito do gênero \textit{Anopheles}. Ela está presente em 22 países 
% americanos, porém as áreas com maior risco de infecção estão localizadas 
% na região amazônica, englobando nove países, e que representaram $68\%$ 
% dos casos de infecção em 2011 $^{[1]}$. Apesar de ser muito comum nas 
% Américas, a malária não é limitada a esse continente, sendo encontrada 
% em países da África e Ásia, tendo resultado em mais de dois milhões de 
% casos de infecção e  445 mil mortes ao redor do mundo em 2016 $^{[2]}$.    
% \\\\
% Notavelmente, a transmissão de doenças por vetores é intimamente relacionada 
% a alterações ambientais que interferem no ecossistema dos organismos 
% transmissores e dos organismos afetados. No caso da Amazônia, povoados 
% agrícolas e agropecuários são alguns dos fatores que mais favorecem a 
% transmissão da doença, tanto pelo desmatamento que causam para seu 
% estabelecimento, quanto pelo agrupamento de pessoas em ambientes 
% próximos ao habitat do vetor $^{[3]}$, em especial por aglomerar migrantes 
% não-imunes próximos a esses criadouros naturais e artificiais $^{[4]}$. 
% \\\\
% Além disso, outros fatores, 
% como chuvas, queimadas e mineração também são muito influentes na 
% transmissão de doenças na região. Esses eventos resultam em perda 
% de habitat, fragmentação de ecossistemas e alterações no clima, 
% afetando a distribuição e abundância de vetores e hospedeiros, bem 
% como a interação entre eles e os patógenos. Ademais, o crescimento 
% populacional e a urbanização também têm um papel importante na disseminação 
% de doenças, uma vez que aumentam a exposição dos seres humanos aos vetores 
% e aos riscos de infecção.
% \\\\
% Diante desse contexto, este trabalho visa investigar a transmissão de 
% doenças por vetores na Amazônia e analisar como os impactos ambientais 
% influenciam a dinâmica de transmissão da malária, os fatores ecológicos 
% e socioeconômicos que afetam essa disseminação e possíveis estratégias 
% de prevenção e controle, tendo como referência principal o Projeto 
% Trajetórias, desenvolvido pelo Centro de Biodiversidade e Serviços Ecossistêmicos (SinBiose/CNPq), que é um dataset incluindo 
% indicadores ambientais, epidemiológicos, econômicos e socioeconômicos 
% para todos os municípios da Amazônia Legal, analisando a relação espacial 
% e temporal entre trajetórias econômicas ligadas à dinâmica dos sistemas 
% agrários, sendo eles rurais de base familiar ou produção agrícola e de 
% gado em larga escala, a disponibilidade de recursos naturais e o risco 
% de doenças $^{[5]}$.

% -----------------------------------
% ELEMENTOS PÓS-TEXTUAIS
% -----------------------------------
\postextual
% ----------------------------------

%\bibliography{biblio}
\printbibliography

%\glossary

% ----------------------------------------------------------
% Apêndices
% ----------------------------------------------------------

% ---
% Inicia os apêndices
% ---
\begin{apendicesenv}

% Imprime uma página indicando o início dos apêndices
\partapendices

\chapter{Resultados Desenvolvidos}

\begin{figure}[h]
\end{figure}	

\end{apendicesenv}
% ---

% ----------------------------------------------------------
% Anexos
% ----------------------------------------------------------

% \begin{anexosenv}

% \partanexos

% \end{anexosenv}
"
%---------------------------------------------------------------------
% ÍNDICE REMISSIVO
%---------------------------------------------------------------------
\phantompart
\printindex

\end{document}