% \chapter{Apêndices}

% A Amazônia é uma das maiores e mais biodiversas florestas tropicais do mundo, 
% abrigando inúmeras espécies de plantas, animais e microrganismos, incluindo 
% vetores e patógenos responsáveis pela transmissão de diversas doenças. Entre 
% elas, uma das mais comuns é a malária, que é causada por protozoários do 
% gênero \textit{Plasmodium}, transmitidos pela picada da fêmea infectada do 
% mosquito do gênero \textit{Anopheles}. Ela está presente em 22 países 
% americanos, porém as áreas com maior risco de infecção estão localizadas 
% na região amazônica, englobando nove países, e que representaram $68\%$ 
% dos casos de infecção em 2011 $^{[1]}$. Apesar de ser muito comum nas 
% Américas, a malária não é limitada a esse continente, sendo encontrada 
% em países da África e Ásia, tendo resultado em mais de dois milhões de 
% casos de infecção e  445 mil mortes ao redor do mundo em 2016 $^{[2]}$.    
% \\\\
% Notavelmente, a transmissão de doenças por vetores é intimamente relacionada 
% a alterações ambientais que interferem no ecossistema dos organismos 
% transmissores e dos organismos afetados. No caso da Amazônia, povoados 
% agrícolas e agropecuários são alguns dos fatores que mais favorecem a 
% transmissão da doença, tanto pelo desmatamento que causam para seu 
% estabelecimento, quanto pelo agrupamento de pessoas em ambientes 
% próximos ao habitat do vetor $^{[3]}$, em especial por aglomerar migrantes 
% não-imunes próximos a esses criadouros naturais e artificiais $^{[4]}$. 
% \\\\
% Além disso, outros fatores, 
% como chuvas, queimadas e mineração também são muito influentes na 
% transmissão de doenças na região. Esses eventos resultam em perda 
% de habitat, fragmentação de ecossistemas e alterações no clima, 
% afetando a distribuição e abundância de vetores e hospedeiros, bem 
% como a interação entre eles e os patógenos. Ademais, o crescimento 
% populacional e a urbanização também têm um papel importante na disseminação 
% de doenças, uma vez que aumentam a exposição dos seres humanos aos vetores 
% e aos riscos de infecção.
% \\\\
% Diante desse contexto, este trabalho visa investigar a transmissão de 
% doenças por vetores na Amazônia e analisar como os impactos ambientais 
% influenciam a dinâmica de transmissão da malária, os fatores ecológicos 
% e socioeconômicos que afetam essa disseminação e possíveis estratégias 
% de prevenção e controle, tendo como referência principal o Projeto 
% Trajetórias, desenvolvido pelo Centro de Biodiversidade e Serviços Ecossistêmicos (SinBiose/CNPq), que é um dataset incluindo 
% indicadores ambientais, epidemiológicos, econômicos e socioeconômicos 
% para todos os municípios da Amazônia Legal, analisando a relação espacial 
% e temporal entre trajetórias econômicas ligadas à dinâmica dos sistemas 
% agrários, sendo eles rurais de base familiar ou produção agrícola e de 
% gado em larga escala, a disponibilidade de recursos naturais e o risco 
% de doenças $^{[5]}$.