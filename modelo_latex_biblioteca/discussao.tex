\chapter{Discussão}

Ao longo do trabalho, foi desenvolvido e analisado um modelo de transmissão 
SIR/SEI para a malária de forma a complementar a metodologia original de Parham \& Michael, 
utilizando dos fatores epidemiológicos da transmissão da doença e complementando com 
dinâmicas de demografia humana e fatores ambientais externos à temperatura e precipitação,
como o desmatamento, considerado no fator multiplicativo das proporções de picadas
causando infecção. 
\\\\
Dado o grande foco alocado na modelagem da transmissão analisando os impactos 
ambientais, sendo necessárias diferentes adaptações do modelo buscando torná-lo o 
mais realista e mais compatível com a dinâmica da doença no ambiente, a análise 
aprofundada dos efeitos de fatores socioeconômicos, assim como a identificação 
de estratégias de prevenção e controle da doença acabaram fugindo ao escopo do 
TCC.
\\\\
Dentro do que foi feito, a principal consideração que se pôde tirar do 
desenvolvimento do trabalho é que esse é um modelo muito sensível aos 
parâmetros utilizados. Pequenas modificações são suficientes para que o modelo 
atinja um equilíbrio ou tenha as populações tendendo a $\pm \infty$.
\\\\
De fato, o que pôde ser especialmente notado quando modificando os parâmetros 
$A, \ B$ e $C$ foi que utilizando os valores indicados
no trabalho de Parham $\&$ Michael \cite{Parham2010} e Eikenberry $\&$ Gummel \cite{OKUNEYE201772},
no caso $A = -0.03, \ B = 1.31, \ C = -4.4$, não aconteceu epidemia.
Iniciando a modelagem com os demais parâmetros utilizados 
na Figura 7, com $k=1$, $\mathcal{R}_0 = 0.0207$, um valor muito abaixo do verificado previamente.
Mesmo com $k=10$, $\mathcal{R}_0 = 0.207$, levando à população de mosquitos à extinção em todos os casos 
e inviabilizando a existência de equilíbrio endêmico.
\\\\
Outro ponto que pôde ser percebido foi que em muitos casos, 
a malária leva muito tempo para entrar em equilíbrio endêmico. Com isso,
apesar de não ter sido possível estudar a aplicação de estratégias de controle da doença,
pode-se chegar à conclusão que medidas de longo prazo podem não ser tão eficazes,
visto que as condições ambientais presentes no início dessa análise
já não seriam as mesmas quando a medida for aplicada.
\\\\
Como trabalho futuro, seria possível comparar a metodologia geral utilizada pelos autores 
referenciados com a metodologia utilizada nesse trabalho, e verificar 
o que poderia ser modificado para que, utilizando os parâmetros originais, o modelo
ainda tivesse um equilíbrio endêmico.
\\\\ 
Ademais, conforme verificado no cálculo dos equilíbrios, algo mais que pode ser 
estudado dando continuidade ao trabalho seria a análise de sua evolução conforme o tempo, visto que eles são dados 
por funções osciladoras, como a taxa de picadas, taxa de nascimentos, de mortalidade, 
a probabilidade de sobrevivência e a taxa de infecção de expostos, que variam dependendo
dos fatores de temperatura e precipitação, conforme apresentado previamente.
