\documentclass[12pt]{article}

\usepackage{setspace}
\usepackage{amsmath,amssymb}
\usepackage{amsfonts}
\usepackage{graphicx}
\usepackage[pdftex,bookmarks=true,bookmarksopen=false,bookmarksnumbered=true,colorlinks=true,linkcolor=black]{hyperref}
\usepackage[utf8]{inputenc}
\usepackage{float}
\usepackage{makecell}
\usepackage{tabularx}
\usepackage{changepage}
\usepackage{pbox}
\usepackage{pdfpages}
\usepackage{mwe}
\usepackage{cellspace}
\setlength\cellspacetoplimit{4pt}
\setlength\cellspacebottomlimit{6pt}

\usepackage[brazil]{babel}
%\usepackage{pstricks}%, egameps}

%\setlength{\textwidth}{17.2cm}
% \setlength{\textheight}{23cm}
%\addtolength{\oddsidemargin}{-22mm} 
%\addtolength{\topmargin}{-15mm} \addtolength{\evensidemargin}{-15mm}
%\setlength{\parskip}{1mm}
%\setlength{\baselineskip}{500mm}

\newtheorem{theorem}{Theorem}[section]
\newtheorem{assumption}{Assumption}
\newtheorem{acknowledgment}{Acknowledgment}
\newtheorem{algorithm}{Algorithm}
\newtheorem{axiom}{Axiom}
\newtheorem{case}{Case}
\newtheorem{claim}{Claim}
\newtheorem{conclusion}{Conclusion} 
\newtheorem{condition}{Condition}
\newtheorem{conjecture}{Conjecture}
\newtheorem{corollary}{Corollary}[section]
\newtheorem{criterion}{Criterion}
\newtheorem{defn}{Definition}[section]

\newtheorem{example}{Example}[section]
\newtheorem{exercise}{Exercise}
\newtheorem{lemma}{Lemma}[section]
\newtheorem{notation}{Notation}
\newtheorem{problem}{Problem}
\newtheorem{proposition}{Proposition}[section]
\newtheorem{remark}{Remark}
\newtheorem{solution}{Solution}
\newtheorem{summary}{Summary}
\newenvironment{proof}[1][Proof]{\textbf{#1.} }{\rule{0.5em}{0.5em}}

\begin{document}

\begin{titlepage}
\begin{center}
\textbf{\LARGE Fundação Getulio Vargas}\\ 
\textbf{\LARGE Escola de Matemática Aplicada}\\
\textbf{\LARGE Curso de Graduação em Matemática Aplicada}

\par
\vspace{160pt}
\textbf{\Large Utilização de indicadores ambientais e epidemiológicos no estudo da dinâmica de doenças transmitidas por vetores}\\
\vspace{80pt}
\textbf{\Large por Raphael Felberg Levy}\\
\end{center}

\par
\vfill
\begin{center}
{{\normalsize Rio de Janeiro - Brasil}\\
{\normalsize \the\year}}
\end{center}
\end{titlepage}

\thispagestyle{empty}

\newpage
\begin{center}
\textbf{\LARGE Fundação Getulio Vargas}\\ 
\textbf{\LARGE Escola de Matemática Aplicada}\\
\textbf{\LARGE Curso de Graduação em Matemática Aplicada}

\par
\vspace{90pt}
\textbf{\Large Utilização de indicadores ambientais e epidemiológicos no estudo da dinâmica de doenças transmitidas por vetores}


\par
\vspace{65pt}
``\textbf{Declaro ser o único autor do presente projeto de
monografia que refere-se ao plano de trabalho a ser executado para continuidade da monografia e ressalto que não recorri a qualquer forma de colaboração ou auxílio de terceiros para realizá-lo a não ser nos casos e para os fins autorizados pelo professor orientador.}''
\end{center}

\par
\vspace{65pt}
\begin{center}


\hrulefill

\vspace{5pt}
\textbf{\Large Raphael Felberg Levy}
\end{center}

\par
\vfill
\begin{center}
{{\normalsize Rio de Janeiro - Brasil}\\
{\normalsize \the\year}}
\end{center}

\thispagestyle{empty}

\newpage
\begin{center}
\textbf{\LARGE Fundação Getulio Vargas}\\ 
\textbf{\LARGE Escola de Matemática Aplicada}\\
\textbf{\LARGE Curso de Graduação em Matemática Aplicada}

\par
\vspace{90pt}
\textbf{\Large Utilização de indicadores ambientais e epidemiológicos no estudo da dinâmica de doenças transmitidas por vetores}


\par
\vspace{65pt}

``\textbf{Projeto de Monografia apresentado à Escola de Matemática Aplicada como requisito parcial para continuidade ao trabalho de monografia.}''
\end{center}

\par
\vspace{65pt}
\begin{center}

\textbf{Aprovado em } \makebox[30pt]{\hrulefill}\textbf{ de }\makebox[120pt]{\hrulefill}\textbf{ de }\makebox[50pt]{\hrulefill}
\\
\vspace{5pt}
\textbf{Grau atribuído ao Projeto de Monografia:} \makebox[30pt]{\hrulefill}\\
\end{center}


\par
\vspace{40pt}
\begin{center}

\hrulefill

\vspace{5pt}
\textbf{Professor Orientador: Flávio Codeço Coelho}\\
\textbf{Escola de Matemática Aplicada}\\
\textbf{Fundação Getúlio Vargas}
\end{center}

\thispagestyle{empty}


\newpage
\tableofcontents
\thispagestyle{empty}

\newpage
\section{Introdução}

A Amazônia é uma das maiores e mais biodiversas florestas tropicais do mundo, abrigando inúmeras espécies de plantas, animais e microrganismos, incluindo vetores e patógenos responsáveis pela transmissão de diversas doenças. Entre elas, uma das mais comuns é a malária, que é causadas por protozoários do gênero \textit{Plasmodium}, transmitidos pela picada da fêmea infectada do mosquito do gênero \textit{Anopheles}. Ela está presente em 22 países americanos, porém as áreas com maior risco de infecção estão localizadas na região amazônica, englobando nove países, e que representaram $68\%$ dos casos de infecção em 2011 $^{[1]}$. Apesar de ser muito comum nas Américas, a malária não é limitada a esse continente, sendo encontrada em países da África e Ásia, tendo resultado em mais de dois milhões de casos de infecção e  445 mil mortes ao redor do mundo em 2016 $^{[2]}$.    
\\\\
Notavelmente, a transmissão de doenças por vetores é intimamente relacionada à alterações ambientais que interferem no ecossistema dos organismos transmissores e dos organismos afetados. No caso da Amazônia, povoados agrícolas e agropecuários são uns dos fatores que mais favorecem a transmissão da doença, tanto pelo desmatamento que causam para seu estabelecimento, assim como o agrupamento de pessoas em ambientes próximos ao habitat do vetor $^{[3]}$. Além disso, outros fatores, como chuvas, queimadas e mineração, também são muito influentes na transmissão de doenças na região. Esses eventos resultam em perda de habitat, fragmentação de ecossistemas e alterações no clima, afetando a distribuição e abundância de vetores e hospedeiros, bem como a interação entre eles e os patógenos. Além disso, o crescimento populacional e a urbanização também têm um papel importante na disseminação de doenças, uma vez que aumentam a exposição dos seres humanos aos vetores e aos riscos de infecção.
\\\\
Diante desse contexto, este trabalho visa investigar a transmissão de doenças por vetores na Amazônia e analisar como os impactos ambientais influenciam a dinâmica de transmissão da malária, os fatores ecológicos e socioeconômicos que afetam essa disseminação e possíveis estratégias de prevenção e controle, tendo como referência principal o Projeto Trajetórias-Sinbiose, elaborado pela FIOCRUZ, um dataset incluindo indicadores ambientais, epidemiológicos, econômicos e socioeconômicos para todos os municípios da Amazônia Legal, analisando a relação espacial e temporal entre trajetórias econômicas ligadas à dinâmica dos sistemas agrários, sendo eles rurais de base familiar ou produção agrícola e de gado em larga escala, a disponibilidade de recursos naturais e o risco de doenças $^{[4]}$.
\\\\
(Ao longo deste trabalho, serão abordados os seguintes tópicos: (1) uma revisão das principais doenças transmitidas por vetores na Amazônia e seus vetores e patógenos associados; (2) análise dos fatores ecológicos, climáticos e socioeconômicos que influenciam a transmissão de doenças; (3) discussão sobre os modelos epidemiológicos, incluindo adaptações aos modelos SIR e SEI, para avaliar o impacto das mudanças ambientais na transmissão de doenças; e (4) identificação de estratégias de prevenção e controle baseadas na compreensão da dinâmica de transmissão e nos desafios específicos da região amazônica.)

\section{Metodologia}

Para a elaboração do trabalho, serão usados dados do dataset do Projeto Trajetórias, e serão abordados métodos de transmissão de doenças baseados em equações diferenciais ordinárias, como o SIR, e, partindo de uma modelagem simples, serão incluídos os fenômenos ambientais, como desmatamento e queimada, para ver como modificações no ecossistema irão interferir no modelo elaborado previamente.

\subsection{Modelos}

Descrevendo primeiramente SIR $^{[5], [6]}$, que pode ser considerado a base de modelos que serão usados ao longo do projeto, este foi desenvolvido por W. O. Kermack e A. G. McKendrick em 1927, sendo um dos modelos mais usados para a modelagem de epidemias, levando em consideração três compartimentos:

\begin{gather*}
    S: \text{número de indivíduos suscetíveis} \\
    I: \text{número de indivíduos infectados} \\
    R: \text{número de indivíduos recuperados}
\end{gather*}
\\
Nesse modelo, os indivíduos saudáveis na classe $S$ são suscetíveis ao contato com indivíduos da classe $I$, e são transferidos para esse compartimento caso contraiam a doença. Indivíduos infectados podem espalhar a doença por contato direto com indivíduos suscetíveis, mas também podem se tornar imunes ao longo do tempo, sendo transferidos para o compartimento $R$. Em geral, $R$ inclui o total de recuperados (imunes) e mortos em decorrência da doença, mas podemos assumir que o número de mortos é muito baixo em relação ao tamanho da população total, podendo ser ignorado. Consideramos também que indivíduos nessa categoria não voltarão a ser suscetíveis ou infecciosos.   
\\\\
Considerando uma epidemia em um espaço curto de tempo e que a doença não é fatal, podemos ignorar dinâmicas vitais de nascimento e morte. Com isso, podemos descrever o modelo SIR através do seguinte sistema de EDOs:

\begin{gather*}
\begin{cases}
\dfrac{dS}{dt} = -\dfrac{\beta SI}{N} \\
\\
\dfrac{dI}{dt} = \dfrac{\beta SI}{N} - \gamma I \\
\\
\dfrac{dR}{dt} = \gamma I
\end{cases}
\end{gather*}
\\
No modelo, $N(t) = S(t)+I(t)+R(t)$, ou seja, a população total no tempo $t$, enquanto que $\beta$ é a taxa de infecção e $\gamma$ é a taxa de recuperação. Dado que $S+I+R$ é sempre constante se ignorarmos nascimento e morte, temos $\dfrac{dS}{dt}+\dfrac{dI}{dt}+\dfrac{dR}{dt} = 0$. 
\\\\
Para que a doença possa se espalhar, é fácil ver que $\dfrac{dI}{dt} = \dfrac{\beta SI}{N} - \gamma I > 0$. Assim, $\dfrac{\beta SI}{N} > \gamma I \Rightarrow \dfrac{\beta S}{N} > \gamma$. Supondo que estamos no início da infeccção, dado que queremos ver como se espalha, $I$ será muito pequeno e $S \approx N$. Concluímos então que $\dfrac{\beta N}{N} > \gamma \Rightarrow \dfrac{\beta}{\gamma} > 1$. É possível derivar esse valor adimensionalizando o modelo: sejam $y^* = \dfrac{S}{N}, \ x^* = \dfrac{I}{N}, \ z^* = \dfrac{R}{N}$ e $t^*=\dfrac{t}{1/\gamma} = \gamma t$, de forma que $y^*+x^*+z^*=1$. Substituindo o sistema de EDOs acima utilizando esses valores:

\begin{gather*}
\begin{cases}
\dfrac{dS}{dt} = \dfrac{d(y^*N)}{d(t^*/\gamma)} = -\dfrac{\beta SI}{N} = -\dfrac{\beta(y^*N)(x^*N)}{N} = -\beta y^*Nx^* \\
\\
\dfrac{dI}{dt} = \dfrac{d(x^*N)}{d(t^*/\gamma)} = \dfrac{\beta SI}{N} - \gamma I = \dfrac{\beta(y^*N)(x^*N)}{N} -\gamma(x^*N) = \beta y^*Nx^* - \gamma x^*N \\
\\
\dfrac{dR}{dt} = \dfrac{d(z^*N)}{d(t^*/\gamma)} = \gamma I = \gamma(x^*N)
\end{cases}
\end{gather*}

Agora, cancelando os fatores $N$ e $\gamma$ em ambos os lados das equações:

\begin{gather*}
\begin{cases}
\dfrac{d(y^*)}{d(t^*)} = -\dfrac{\beta y^*x^*}{\gamma} \\
\\
\dfrac{d(x^*)}{d(t^*)} = \dfrac{\beta y^*x^*}{\gamma} - x^* \\
\\
\dfrac{d(z^*)}{d(t^*)} = x^*
\end{cases}
\end{gather*}
\\
Sendo assim temos um sistema dado apenas por $y^*$ e $x^*$ e o parâmetro $\dfrac{\beta}{\gamma}$, que podemos chamar de $R_0$.
\\\\
Como esse trabalho será focado principalmente na modelagem de malária, irei agora apresentar um dos primeiros modelos desenvolvidos especialmente para essa doença, por Sir Ronald Ross em 1911 $^{[7]}$, que usa duas EDOs distintas das apresentadas acima:

\begin{gather*}
\begin{cases}
\dfrac{dI}{dt} = bp'i\dfrac{N-I}{N} -aI\\
\\
\dfrac{di}{dt} = bp(n-i)\dfrac{I}{N} - mI
\end{cases}
\end{gather*}
\\
Nesse caso, $N$ é a população humana total, $I(t)$ é o número de humanos infectados no tempo $t$, $n$ é a população total de mosquitos, $i(t)$ é o número de mosquitos infectados no tempo $t$, $b$ é a taxa de mordidas, $p$ é a probabilidade de transmissão do humano para o mosquito por mordida, $p'$ é a probabilidade de transmissão do mosquito para o humano por mordida, $a$ é a taxa de recuperação da infecção de um humano e $m$ é a taxa de mortalidade dos mosquitos. $bp'i\dfrac{N-I}{N}dt -aIdt$ representam respectivamente o número de novos humanos infectados e o número de humanos recuperados no intervalo $dt$, enquanto que $bp(n-i)\dfrac{I}{N}dt - mIdt$ representam respectivamente o número de novos mosquitos infectados e o número de mosquitos que morrem nesse intervalo de tempo, assumindo que a infeccção não interfere na taxa de mortalidade dos mosquitos.
\\\\
Para esse modelo, Ross discutiu dois pontos de equilíbrio, em que $\dfrac{dI}{dt} = \dfrac{di}{dt} = 0$. Eles ocorrem quando $I=i=0$, que é o caso onde não existe malária, e, para $I, i > 0$, $I = N\dfrac{1-amN/(b^2pp'n)}{1+aN/(bp'n)}$ e $i = n\dfrac{1-amN/(b^2pp'n)}{1+m/(bp)}$. Ainda, para que a doença se estabeleça, $n$ deve ser maior que um valor limiar $n^* = \dfrac{amN}{b^2pp'}$. Nesse caso a doença se torna endêmica. Caso $n<n^*$, o equilíbrio estará em $I=i=0$ e a dença irá desaparecer.
\\\\
Dividindo as equações dos pontos de equilíbrio por $I \times i$, temos:

\begin{gather*}
\begin{cases}
\dfrac{bp}{N} = \dfrac{bpn}{Ni} -\dfrac{m}{I} \\
\\
\dfrac{bp'}{N} = \dfrac{bp'}{I} -\dfrac{a}{i} 
\end{cases}
\end{gather*}
\\
O que transforma o problema em um sistema linear com dois desconhecidos, $I$ e $i$.
\\\\
Agora, irei apresentar o modelo que será usado para o desenvolvimento do trabalho, elaborado por Paul E. Parham e Edwin Michael em 2010, que leva em consideração fatores como a chuva e temperatura ($R$ e $T$, respectivamente) $^{[8]}$. 
\\\\
Definindo as equações que serão utilizadas:
\begin{gather*}
\begin{cases}
\dfrac{dS_M}{dt} = b - ab_1\bigg(\dfrac{I_H}{N}\bigg)S_M - \mu S_M\\
\\
\dfrac{dE_M}{dt} = ab_1\bigg(\dfrac{I_H}{N}\bigg)S_M - \mu E_M - ab_1\bigg(\dfrac{I_H(t-\tau_M)}{N}\bigg)S_M(t-\tau_M)l(\tau_M)\\
\\
\dfrac{dI_M}{dt} = ab_1\bigg(\dfrac{I_H(t-\tau_M)}{N}\bigg)S_M(t-\tau_M)l(\tau_M) -\mu I_M\\
\\
\dfrac{dS_H}{dt} = -ab_2\bigg(\dfrac{I_M}{N}\bigg)S_H\\
\\
\dfrac{dI_H}{dt} = ab_2\bigg(\dfrac{I_M(t-\tau_H)}{N}\bigg)S_H(t-\tau_H)-\gamma I_H\\
\\
\dfrac{dR_H}{dt} = \gamma I_H
\end{cases}
\end{gather*}
\\\\
Tendo as equações do modelo, irei primeiro definir os parâmetros utilizados na modelagem e outras funções necessárias, e depois as variáveis usadas:
\begin{adjustwidth}{-2cm}{}
\begin{center}
\renewcommand{\arraystretch}{1.5}
\begin{tabular}{|c | l | c|} 
 \hline
 \raisebox{-1ex}{\textbf{Parâmetro}} & \raisebox{-1ex}{\textbf{Definição}} & \raisebox{-1ex}{\textbf{Cálculo}}\\ 
 \hline
 $T(t)$ & \pbox{8cm}{\rule{0pt}{4.5ex}Temperatura\rule[-2.5ex]{0pt}{0pt}} & $T_1 (1 + T_2 \cos(\omega_1t - \phi_1))$\\ 
 \hline
 $R(t)$ & \pbox{8cm}{\rule{0pt}{4.5ex}Precipitação\rule[-2.5ex]{0pt}{0pt}} & $R_1 (1 + R_2 \cos(\omega_2t - \phi_2))$ \\
 \hline
 $b(R, T)$ & \pbox{8cm}{\rule{0pt}{4.5ex}Taxa de nascimento de mosquitos (/ dia)\rule[-2.5ex]{0pt}{0pt}} & $\dfrac{B_E  p_E(R)  p_L(R,T)  p_P(R)}{(\tau_E + \tau_L(T) + \tau_P)}$\\ 
 \hline
 $a(T)$ & \pbox{8cm}{\rule{0pt}{4.5ex}Taxa de mordidas (/dia)\rule[-2.5ex]{0pt}{0pt}} & $\dfrac{(T - T_1)}{D_1}$ \\
 \hline
 $\mu(T)$ & \pbox{8cm}{\rule{0pt}{3ex}Taxa de mortalidade de mosquitos per capita (/ dia)\rule[-1.5ex]{0pt}{0pt}} & $-\log(p(T))$ \\
 \hline
 $\tau_M(T)$ & \pbox{8cm}{\rule{0pt}{4.5ex}Duração do ciclo de esporozoitos (dias)\rule[-2.5ex]{0pt}{0pt}} & $\dfrac{DD}{(T - T_{min})}$ \\
 \hline
 $\tau_L(T)$ & \pbox{8cm}{\rule{0pt}{4.5ex}Duração da fase de desenvolvimento das larvas (dias)\rule[-2.5ex]{0pt}{0pt}} & $\dfrac{1}{c_1T + c_2}$ \\
 \hline
 $p(T)$ & \pbox{8cm}{\rule{0pt}{3ex}Taxa diária de sobrevivência dos mosquitos (dias)\rule[-1.5ex]{0pt}{0pt}} & $e^{(-1 / (AT^2 + BT + C))}$ \\
 \hline
 $p_L(R)$ & \pbox{8cm}{\rule{0pt}{3ex}Probabilidade de sobrevivência das larvas dependente de chuva\rule[-1.5ex]{0pt}{0pt}} & $(\dfrac{4p_{ML}}{R_L^2})R(R_L - R)$ \\
 \hline
 $p_L(T)$ & \pbox{8cm}{\rule{0pt}{3ex}Probabilidade de sobrevivência das larvas dependente de temperatura\rule[-1.5ex]{0pt}{0pt}} & $e^{-(c_1T + c_2)}$ \\
 \hline
 $p_L(R, T)$ & \pbox{8cm}{\rule{0pt}{3ex}Probabilidade de sobrevivência das larvas dependente de temperatura e chuva\rule[-1.5ex]{0pt}{0pt}} & $p_L(R)p_L(T)$ \\
 \hline
 $l(\tau_M)(T)$ & \pbox{8cm}{\rule{0pt}{3ex}Probabilidade de sobrevivência de mosquitos durante o ciclo de esporozoitos (/ dia)\rule[-1.5ex]{0pt}{0pt}} & $p(T)^{\tau_M(T)}$ \\
 \hline
 $M(t)$ & \pbox{8cm}{\rule{0pt}{3ex}Número total de mosquitos\rule[-1.5ex]{0pt}{0pt}} & $S_M(t) + E_M(t) + I_M(t)$ \\
 \hline
 $N(t)$ & \pbox{8cm}{\rule{0pt}{3ex}Número total de humanos\rule[-1.5ex]{0pt}{0pt}} & $S_H(t) + I_H(t) + R_H(t)$ \\  
 \hline
\end{tabular}
\end{center}
\end{adjustwidth}

\begin{adjustwidth}{-2cm}{}
\begin{center}
\renewcommand{\arraystretch}{1.5}
\begin{tabular}{|c | c|} 
 \hline
 \textbf{Parâmetro} & \textbf{Definição}\\ 
 \hline
 $b_1$ & \makecell[l]{\rule{0pt}{3ex}Proporção de mordidas de mosquitos suscetíveis \\ em humanos infectados que produzem infecção\rule[-1.5ex]{0pt}{0pt}} \\
 \hline
 $b_2$ & \makecell[l]{\rule{0pt}{3ex}Proporção de mordidas de mosquitos infectados \\ em humanos suscetíveis que produzem infecção\rule[-1.5ex]{0pt}{0pt}} \\
 \hline
 $\tau_H$ & \makecell[l]{\rule{0pt}{3ex}Período latente da infecção em humanos (dias)\rule[-1.5ex]{0pt}{0pt}} \\
 \hline
 $\gamma$ & \makecell[l]{\rule{0pt}{3ex}1/Duração média da infecciosidade em humanos (dias$^{-1}$)\rule[-1.5ex]{0pt}{0pt}} \\
 \hline
 $T_1$ & \makecell[l]{\rule{0pt}{3ex}Temperatura média na ausência de sazonalidade ($^\circ C$)\rule[-1.5ex]{0pt}{0pt}} \\
 \hline
 $T_2$ & \makecell[l]{\rule{0pt}{3ex}Amplitude da variabilidade sazonal na temperatura\rule[-1.5ex]{0pt}{0pt}} \\
 \hline
 $R_1$ & \makecell[l]{\rule{0pt}{3ex}Precipitação mensal média na ausência de \\ sazonalidade (mm)\rule[-1.5ex]{0pt}{0pt}} \\
 \hline
 $R_2$ & \makecell[l]{\rule{0pt}{3ex}Amplitude da variabilidade sazonal na precipitação\rule[-1.5ex]{0pt}{0pt}} \\
 \hline
 $\omega_1$ & \makecell[l]{\rule{0pt}{3ex}Frequência angular das oscilações sazonais na temperatura (meses$^{-1}$)\rule[-1.5ex]{0pt}{0pt}} \\
 \hline
 $\omega_2$ & \makecell[l]{\rule{0pt}{3ex}Frequência angular das oscilações sazonais na precipitação (meses$^{-1}$)\rule[-1.5ex]{0pt}{0pt}} \\
 \hline
 $\phi_1$ & \makecell[l]{\rule{0pt}{3ex}``Phase lag" da variabilidade da temperatura (defasagem de fase)\rule[-1.5ex]{0pt}{0pt}} \\
 \hline
 $\phi_2$ & \makecell[l]{\rule{0pt}{3ex}``Phase lag" da variabilidade da precipitação (defasagem de fase)\rule[-1.5ex]{0pt}{0pt}} \\
 \hline
 $B_E$ & \makecell[l]{\rule{0pt}{3ex}Número de ovos colocados por adulto por oviposição\rule[-1.5ex]{0pt}{0pt}} \\
 \hline
 $p_{ME}$ & \makecell[l]{\rule{0pt}{3ex}Probabilidade de sobrevivência dos ovos\rule[-1.5ex]{0pt}{0pt}} \\
 \hline
 $p_{ML}$ & \makecell[l]{\rule{0pt}{3ex}Probabilidade de sobrevivência das larvas\rule[-1.5ex]{0pt}{0pt}} \\
 \hline
 $p_{MP}$ & \makecell[l]{\rule{0pt}{3ex}Probabilidade de sobrevivência das pupas\rule[-1.5ex]{0pt}{0pt}} \\
 \hline
 $\tau_E$ & \makecell[l]{\rule{0pt}{3ex}Duração da fase de desenvolvimento dos ovos (dias)\rule[-1.5ex]{0pt}{0pt}} \\
 \hline
\end{tabular}
\end{center}
\end{adjustwidth}


\begin{adjustwidth}{-2cm}{}
\begin{center}
\renewcommand{\arraystretch}{1.5}
\begin{tabular}{|c | c|} 
 \hline
 \textbf{Parâmetro} & \textbf{Definição}\\ 
 \hline
  $\tau_P$ & \makecell[l]{\rule{0pt}{3ex}Duração da fase de desenvolvimento das pupas (dias)\rule[-1.5ex]{0pt}{0pt}} \\
 \hline
 $R_L$ & \makecell[l]{\rule{0pt}{3ex}Chuva limite até que os sítios de reprodução sejam eliminados, \\ removendo indivíduos de estágio imaturo (mm)\rule[-1.5ex]{0pt}{0pt}} \\
 \hline
 $T_{min}$ & \makecell[l]{\rule{0pt}{3ex}Temperatura mínima, abaixo dessa temperatura não há desenvolvimento \\ do parasita ($^\circ C$)\rule[-1.5ex]{0pt}{0pt}} \\
 \hline
 $DD$ & \makecell[l]{\rule{0pt}{3ex}``Degree days" para desenvolvimento do parasita $^{[9]} (^\circ C \ \text{dias})$\rule[-1.5ex]{0pt}{0pt}} \\
 \hline
 $A$ & \makecell[l]{\rule{0pt}{3ex}Constante: -0.03 ($^\circ C^2 \ \text{dias}^{-1}$)\rule[-1.5ex]{0pt}{0pt}} \\
 \hline
 $B$ & \makecell[l]{\rule{0pt}{3ex}Constante: 1.31 ($^\circ C \ \text{dias}^{-1}$)\rule[-1.5ex]{0pt}{0pt}} \\
 \hline
 $C$ & \makecell[l]{\rule{0pt}{3ex}Constante: -4.4 ($\text{dias}^{-1}$)\rule[-1.5ex]{0pt}{0pt}} \\
 \hline
 $D_1$ & \makecell[l]{\rule{0pt}{3ex}Constante: 36.5 ($^\circ C \ \text{dias}$)\rule[-1.5ex]{0pt}{0pt}} \\
 \hline
 $c_1$ & \makecell[l]{\rule{0pt}{3ex}Constante: 0.00554 ($^\circ C \ \text{dias}^{-1}$)\rule[-1.5ex]{0pt}{0pt}} \\
 \hline
 $c_2$ & \makecell[l]{\rule{0pt}{3ex}Constante: -0.06737 ($\text{dias}^{-1}$)\rule[-1.5ex]{0pt}{0pt}} \\
 \hline
\end{tabular}
\end{center}
\end{adjustwidth}
\\
\\
Com essas funções e parâmetros, podemos calcular o número reprodutivo básico ($R_0$) e os equilíbrios endêmicos:
\begin{gather*}
R_0 = \dfrac{Ma^2b_1b_2l(\tau_M)}{\gamma \mu(T)N} \\
\\
I_M^* = \dfrac{M(R_0-1)}{(\dfrac{R_0}{l(\tau_M)}) + (\dfrac{ab_2M}{\gamma N})} \\
\\
I_H^* = \dfrac{N(R_0-1)}{R_0 + (\dfrac{ab_1}{\mu})}
\end{gather*}
\newpage
Tendo as equações e parâmetros, a modelagem foi feita inicialmente utilizando dados da zona rural de Manaus, no período de 2004 a 2008, que foram selecionados devido à maior incidência de casos de malária causados por $P. \ vivax$. Usando a função de incidência utilizada no projeto Trajetorias $^{[4]}$, temos:
\begin{gather*}
    \text{Inc}(d, m, z, t_1, t_2) = \dfrac{\text{Casos}(d, m, z, t_1, t_2)}{\text{Pop}(m,z,(t_1+t_2)/2) * 5 \ \text{anos}} * 10^5,
\end{gather*}
onde $\text{casos}(d, m, z, t_1, t_2)$ é o número de casos da doença $d$ na zona $z$ do município $m$, e $t_1$ e $t_2$ são os anos iniciais e finais do intervalo, enquanto que $\text{pop}(m,z,(t_1+t_2)/2) * 5 \ \text{anos}$ é a população na zona $z$ do município $m$ no meio do período multiplicado pelo total de anos de observação. Nesse caso, poderíamos indicar como:
\begin{gather*}
    \text{Inc}(\text{Vivax}, \text{Manaus}, \text{Rural}, 2004, 2008) = \dfrac{\text{Casos}(\text{Vivax}, \text{Manaus}, \text{Rural}, 2004, 2008)}{\text{Pop}(\text{Manaus}, \text{Rural}, 2006) * 5 \ \text{anos}} * 10^5  \\\\
    184030.8 = \dfrac{78745}{5\text{Pop}} * 10^5 \Rightarrow Pop \approx 8558
\end{gather*}
Usando dados da população total de Manaus nesse período, cuja incidência foi de 3106.4 e o número de casos foi de 262264, a população total do município foi estimada como sendo de 1688540 habitantes. Com isso, a população rural pôde ser considerada como aproximadamente 0.5$\%$ da população do município.
\\\\
Tendo estimado o tamanho porcentual da população rural na cidade, foi possível calcular essa população em cada um dos anos da análise através de uma interpolação linear feita com dados de séries históricas do IBGE $^{[10]}$.

\newpage
\subsubsection{yy}

yy

% \[\left\{ 
\begin{align*}
f_{i}(x) = (10x + 100), \qquad \text{(1)} \tag{1}\\
f_{ii}(x) = (20x + 200), \qquad \text{(2)} \tag{2} \\
f_{iii}(x) = (30x + 300), \qquad \text{(3)} \tag{3} \\
\end{align*} %\right.\]

xx

\begin{align*}
Vm_{i}(p,l) = ((-1.9141)p + 49.466)l + ((199.51)p - 10795.0), \text {$l$=0} \tag{4}
\end{align*}

\begin{align*}
f_{n}(y) = \frac{y}{1000}, \tag{5}
\end{align*}

\subsection{xx}

xx

%\[\left\{
\begin{align*} 
Funcao_{i}(p) = \gamma + \delta p + \theta p^2 + \omega p^3, \qquad \qquad \qquad \qquad \qquad \qquad \qquad \qquad \text{(6)} \tag{6} \\
\end{align*} %\right.\]

\newpage
\section{Resultados}

Nesta seção serão apresentados os resultados esperados...

\newpage
\section{Discussão}

\newpage
\section{Conclusão}


\newpage

\section{Referências Cronológicas}
[1] Pimenta et al. An overview of malaria transmission from the perspective of Amazon $\text{Anopheles vectors}$. Mem Inst Oswaldo Cruz, Rio de Janeiro, Vol. 110(1): 23-47, February 2015. https://doi.org/10.1590/0074-02760140266.  

\noindent [2] Josling, G. A., Williamson, K. C., Llinás, M. Regulation of Sexual Commitment and Gametocytogenesis in Malaria Parasites. Annual Review of Microbiology 2018 72:1, 501-519. https://doi.org/10.1146/annurev-micro-090817-062712. 

\noindent [3] Silva-Nunes, M., Codeço, C. T. et al. Malaria on the Amazonian Frontier: Transmission Dynamics, Risk Factors, Spatial Distribution, and Prospects for Control. Am J Trop Med Hyg. 2008 Oct;79(4):624-35. PMID: 18840755.

\noindent [4] Rorato, A.C., Dal’Asta, A.P., Lana, R.M. et al. Trajetorias: a dataset of environmental, epidemiological, and economic indicators for the Brazilian Amazon. Sci Data 10, 65 (2023). https://doi.org/10.1038/s41597-023-01962-1.

\noindent [5] Coelho, F. C. Github Modelagem-Matematica-IV. https://github.com/fccoelho/Modelagem-Matematica-IV/tree/master

\noindent [6] Prasad, R., Sagar, S.K., Parveen, S. et al. Mathematical modeling in perspective of vector-borne viral infections: a review. Beni-Suef Univ J Basic Appl Sci 11, 102 (2022). https://doi.org/10.1186/s43088-022-00282-4.

\noindent [7] Bacaër, N. (2011). Ross and malaria (1911). In: A Short History of Mathematical Population Dynamics. Springer, London. https://doi.org/10.1007/978-0-85729-115-8$\_$12.

\noindent [8] Parham, P.E., Michael, E. (2010). Modelling Climate Change and Malaria Transmission. In: Michael, E., Spear, R.C. (eds) Modelling Parasite Transmission and Control. Advances in Experimental Medicine and Biology, vol 673. Springer, New York, NY. https://doi.org/10.1007/978-1-4419-6064-1$\_$13.

\noindent [9] McCord, G. Malaria ecology and climate change. Eur. Phys. J. Spec. Top. 225, 459–470 (2016). https://doi.org/10.1140/epjst/e2015-50097-1

\noindent [10] Censo - Séries históricas. Brasil / Amazonas / Manaus. 
\\https://cidades.ibge.gov.br/brasil/am/manaus/pesquisa/43/0?tipo=grafico

\newpage
\section{Referências Usadas}

[1] Rorato, A.C., Dal’Asta, A.P., Lana, R.M. et al. Trajetorias: a dataset of environmental, epidemiological, and economic indicators for the Brazilian Amazon. Sci Data 10, 65 (2023). https://doi.org/10.1038/s41597-023-01962-1.

\noindent [2] Prasad, R., Sagar, S.K., Parveen, S. et al. Mathematical modeling in perspective of vector-borne viral infections: a review. Beni-Suef Univ J Basic Appl Sci 11, 102 (2022). https://doi.org/10.1186/s43088-022-00282-4.

\noindent [3] Vyhmeister E., Provan G., Doyle B., Bourke B. Multi-cluster and environmental dependant vector born disease models. Heliyon, Volume 6, Issue 9 (2020). https://doi.org/10.1016/j.heliyon.2020.e04090

\noindent [4] Arquam, M., Singh, A., Cherifi, H. (2020). Integrating Environmental Temperature Conditions into the SIR Model for Vector-Borne Diseases. In: Cherifi, H., Gaito, S., Mendes, J., Moro, E., Rocha, L. (eds) Complex Networks and Their Applications VIII. COMPLEX NETWORKS 2019. Studies in Computational Intelligence, vol 881. Springer, Cham. https://doi.org/10.1007/978-3-030-36687-2\_34.

\noindent [5] ALVES, Leon Diniz. Weather-driven mathematical models of dengue transmission dynamics in twelve Brazilian sites. 2021. 137 f. Tese (Doutorado em Biologia Computacional e Sistemas) - Instituto Oswaldo Cruz, Fundação Oswaldo Cruz, Rio de Janeiro, 2021. https://www.arca.fiocruz.br/handle/icict/52536	.

\noindent [6] Alves, Leon Diniz, Raquel Martins Lana, and Flávio Codeço Coelho. 2021. ``A Framework for Weather-Driven Dengue Virus Transmission Dynamics in Different Brazilian Regions" International Journal of Environmental Research and Public Health 18, no. 18: 9493. https://doi.org/10.3390/ijerph18189493.

\noindent [7] Prasad, R., Sagar, S.K., Parveen, S. et al. Mathematical modeling in perspective of vector-borne viral infections: a review. Beni-Suef Univ J Basic Appl Sci 11, 102 (2022). https://doi.org/10.1186/s43088-022-00282-4.

\noindent [8] Abdullah, Seadawy, A. & Jun, W. New mathematical model of vertical transmission and cure of vector-borne diseases and its numerical simulation. Adv Differ Equ 2018, 66 (2018). 
https://doi.org/10.1186/s13662-018-1516-z.

\noindent [9] N. Shah and J. Gupta, "SEIR Model and Simulation for Vector Borne Diseases," Applied Mathematics, Vol. 4 No. 8A, 2013, pp. 13-17. doi: 10.4236/am.2013.48A003.

\end{document}