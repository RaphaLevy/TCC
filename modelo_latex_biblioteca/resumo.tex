\setlength{\absparsep}{18pt} 
\begin{resumo}[Resumo]
    A malária é uma doença infecciosa transmitida por mosquitos infectados por protozoários do gênero \textit{Plasmodium}, sendo a região amazônica considerada área endêmica 
    para a doença. Esse trabalho tem como intuito analisar o comportamento dessa transmissão baseado em modificações climáticas e ambientais, como temperatura, precipitação e desmatamento, 
    através de modificações propostas aos modelos SIR e SEI, de forma a contribuir no estudo de aplicações 
    de efeitos externos na evolução da doença. 
    O Projeto Trajetórias, desenvolvido pelo Centro de Biodiversidade e Serviços Ecossistêmicos (SinBiose/CNPq), será usado como base de referência para as análises.
    

 Palavras-chave: Modelagem biológica. Malária. Amazônia. SIR. SEI.
\end{resumo}

\begin{resumo}[Abstract]
 \begin{otherlanguage*}{english}
    Malaria is an infectious disease transmitted by mosquitoes infected by protozoa of the genus \textit{Plasmodium}, with the Amazon region being considered an endemic area 
    for the disease. This work aims to analyze the behavior of this transmission based on climatic and environmental changes, such as temperature, precipitation and deforestation, through proposed 
    modifications to the SIR and SEI models, in order to contribute to the study of applications
    of external effects on the evolution of the disease. 
    The Trajetórias Project, developed by the Synthesis Center 
    on Biodiversity and Ecosystem Services (SinBiose/CNPq) will be used as a reference base for the analyses.
    
 \end{otherlanguage*}

 Keywords: Biological modelling. Malaria. Amazon. SIR. SEI.
\end{resumo}